% Options for packages loaded elsewhere
\PassOptionsToPackage{unicode}{hyperref}
\PassOptionsToPackage{hyphens}{url}
%
\documentclass[
]{report}
\usepackage{amsmath,amssymb}
\usepackage{lmodern}
\usepackage{iftex}
\ifPDFTeX
  \usepackage[T1]{fontenc}
  \usepackage[utf8]{inputenc}
  \usepackage{textcomp} % provide euro and other symbols
\else % if luatex or xetex
  \usepackage{unicode-math}
  \defaultfontfeatures{Scale=MatchLowercase}
  \defaultfontfeatures[\rmfamily]{Ligatures=TeX,Scale=1}
\fi
% Use upquote if available, for straight quotes in verbatim environments
\IfFileExists{upquote.sty}{\usepackage{upquote}}{}
\IfFileExists{microtype.sty}{% use microtype if available
  \usepackage[]{microtype}
  \UseMicrotypeSet[protrusion]{basicmath} % disable protrusion for tt fonts
}{}
\makeatletter
\@ifundefined{KOMAClassName}{% if non-KOMA class
  \IfFileExists{parskip.sty}{%
    \usepackage{parskip}
  }{% else
    \setlength{\parindent}{0pt}
    \setlength{\parskip}{6pt plus 2pt minus 1pt}}
}{% if KOMA class
  \KOMAoptions{parskip=half}}
\makeatother
\usepackage{xcolor}
\setlength{\emergencystretch}{3em} % prevent overfull lines
\providecommand{\tightlist}{%
  \setlength{\itemsep}{0pt}\setlength{\parskip}{0pt}}
\setcounter{secnumdepth}{-\maxdimen} % remove section numbering
\ifLuaTeX
  \usepackage{selnolig}  % disable illegal ligatures
\fi
\usepackage[]{natbib}
\bibliographystyle{plainnat}
\IfFileExists{bookmark.sty}{\usepackage{bookmark}}{\usepackage{hyperref}}
\IfFileExists{xurl.sty}{\usepackage{xurl}}{} % add URL line breaks if available
\urlstyle{same} % disable monospaced font for URLs
\hypersetup{
  hidelinks,
  pdfcreator={LaTeX via pandoc}}

\author{}
\date{}

\begin{document}

Title: PhD Thesis Author: R.L. Trotta

\markedchapter{Introduction}{Test Chapter}\label{Chapter-1}

\hypertarget{Section-1.1}{%
\section{Test}\label{Section-1.1}}

This is outlines how to setup each section.

\hypertarget{subsection-1}{%
\subsection{Subsection 1}\label{subsection-1}}

Some text\ldots{}

\hypertarget{subsection-2}{%
\subsection{Subsection 2}\label{subsection-2}}

Some more text\ldots{}

\hypertarget{subsubsection-1}{%
\subsubsection{Subsubsection 1}\label{subsubsection-1}}

The key is to have a space before and after each (sub)subsection.

\hypertarget{subsubsection-2}{%
\subsubsection{Subsubsection 2}\label{subsubsection-2}}

\begin{verbatim}
But make sure you don’t indent otherwise things will get weird. As you can see here it cuts it all off and can’t be read. Obviously, this is unwanted.
\end{verbatim}

\hypertarget{subsubsection-3}{%
\subsubsection{Subsubsection 3}\label{subsubsection-3}}

But here you can see that I did not indent and things don't get weird.
The lines no longer are cut off. I am not sure why this happens.
Begin\{verbatim\} is called whenever the indent is explicit in
Scrivener.

Also, add a space between paragraphs.
\markedchapter{Introduction}{Introduction}\label{Chapter-2}

\begin{quote}
\centering \emph{"To a man devoid of blinders, there is no finer sight than that of intelligence at grips with a reality that transcends it."}\\
\centering \emph{- Albert Camus, Myth of Sisyphus}
\end{quote}

\hfill

The meaning of life and existence has fascinated humanity for all of
written history. Starting with the foundations of mathematics and
philosophy by the ancient Greeks through the time of enlightenment, the
rationality driven by the materialistic world has shaped our current
society.~ \textless Describe the close relation of physics and
philosophy and how the two became so separated throughout the modern
era.\textgreater{}

Society in the 21st century is much different now. The modern era of
philosophy was~ coming to an end and a new era of post-modern
hyperreality started. (Dostoevsky said, ``if God is dead, then
everything is permitted'') Beginning~with Fredrick Nietzsche's famous
``God is dead'', the next 60 years was categorized years by a dichotomy
of profound scientific and unprecedented tragedy. Foundations of science
and society had been shattered and needed to be built anew. The theories
of general relativity and quantum mechanics rewrote the textbooks, just
as war unlike any ever seen were coming ahead. These two fronts collided
in the creation of nuclear weapons which provided countries with weapons
that hitherto been inconceivable.

The world changed forever in this moment. This was the moment where

\hypertarget{Section-2.1}{%
\section{Historical Context}\label{Section-2.1}}

\markedchapter{Theory}{Theory}\label{Chapter-3}

\hypertarget{Section-3.1}{%
\section{Quantum Chromodynamics}\label{Section-3.1}}

See section III.B on Medium energy nuclear physics (page 10):
{[}https://arxiv.org/pdf/2112.02309.pdf{]}.

The relevant part is the description of the correlation functions as
objects that need to be extracted from data. You may be able to use
something along the lines of this language to explain the PDF/GPDs in
your thesis.

\hypertarget{Section-3.2}{%
\section{Accessing Hadron Structure}\label{Section-3.2}}

\hypertarget{subsection}{%
\subsection{Subsection}\label{subsection}}

\hypertarget{subsubsection}{%
\subsubsection{Subsubsection}\label{subsubsection}}

\hypertarget{Section-3.3}{%
\section{Parton Distribution Functions and Form
Factors}\label{Section-3.3}}

\hypertarget{model-descriptions}{%
\subsection{Model Descriptions}\label{model-descriptions}}

\hypertarget{Section-3.4}{%
\section{Generalized Parton Distributions and Transverse Momentum
Distributions}\label{Section-3.4}}

Be brief on TMD section.

\hypertarget{model-descriptions-1}{%
\subsection{Model Descriptions}\label{model-descriptions-1}}

\hypertarget{Section-3.5}{%
\section{Experimental Considerations}\label{Section-3.5}}

Sullivan Process and Off-Shell Considerations

\hypertarget{Section-3.6}{%
\section{Kinematic Overview}\label{Section-3.6}}

Table of kinematics and purpose of each \(Q^2\) value (\(Q^2\), \(W\),
\(x\), \(t\)). Include phase space plots.
\markedchapter{Results and Discussion}{Measurements at Jlab Hall C}\label{Chapter-4}

\hypertarget{Section-4.1}{%
\section{Overview}\label{Section-4.1}}

\hypertarget{Section-4.2}{%
\section{Accelerator}\label{Section-4.2}}

Since 1995, the Continuous Wave Electron Accelerator Facility (CEBAF) at
Jlab has been cornerstone to medium energy nuclear research. CEBAF uses
a high intensity continuous wave (CW) beam to deliver electrons to four
experimental halls (Halls A, B, C, and D). In reality, this CW beam is
not truly continuous, rather contains an intrinsic microstructure of
\textasciitilde2 ps short beam pulses that occur at a fundamental
frequency (\(f_0\)) of 1497 MHz \cite{reece_continuous_2016}. This is a
result of the Radio-Frequency (RF) power used in the SRF resonate
cavities which allows for four sequential electron bunches that are
subsequently sent to the four halls.

\begin{Mfigure}{Schematic of CEBAF 12 GeV Upgrade}
  \centering
  \includegraphics[width=0.75\linewidth]{cebaf_12gev.pdf}
  \caption{Schematic of CEBAF 12 GeV Upgrade.}
  \label{fig:2-1_cebaf12gev}
\end{Mfigure}



Each electron bunch is sent into the injector beamline, where they are
accelerated anywhere between 67 to 123 MeV, depending on the desired
beam energy \cite{pilat_12_2012}. From here, they are sent to the north
linac where they are accelerated further by 1.1 GeV. The beam is then
steered by the east arc into the south linac where they gain an
additional 1.1 GeV. Finally, the beam is steered back to the north linac
by the west arc where it can repeat this cycle. The beam can be
recirculated up to a total of five times, where each recirculation is
known as a pass. These passes correspond to the following beam energies:
1-pass (2.2 GeV), 2-pass (4.4 GeV), 3-pass (6.6 GeV), 4-pass (8.8 GeV),
5-pass (12.1 GeV).

Once the desired beam energy is obtained, it can be diverted to the
halls by using separators which are located at the end of the south
linac. In the 6 GeV era, a photo-cathode electron gun used three lasers
pulsing at 500 MHz (i.e.~\(f_0\)/3) which were eventually separated and
directed to each respective hall \cite{kazimi_operational_2019}. After
the 12 GeV upgrade, which includes the addition of Hall D, there had to
be a new beam pattern constructed in order to allow simultaneous beam in
all four halls. This new pattern included modifications to the injector
system and the RF separator extraction system. The injector system added
a fourth laser as well as a new 250 MHz pulse rate while the RF
extraction system had less straightforward changes as a pass-dependent
fix was implemented. For lower passes and when only Halls A, B, C are at
the highest passes, the laser pulses remained at 500 MHz. In the
situation where all four halls are running at the highest pass
(i.e.~5-pass), they are operating at 250 MHz. To allow this a new
separator called the ``5th pass separator'', was added which operates at
750 MHz. This ``5th pass separator'' (see figure
\ref{fig:2-1_cebaf12gev}) sends the separated beam around the 10th west
arc to Hall D.

\hypertarget{Section-4.3}{%
\section{Hall C Beam Line}\label{Section-4.3}}

Halls C accepts the beam through a long pipe that starts at the Beam
Switch Yard (BSY) and ends at the transport line
\cite{sta_jeerson_2019}. In order to reach the hall, the beam is bent in
the Hall C arc (see figure \ref{fig:2-2_hallc_arc}) using a series of
eight dipole magnets. From there, it enters the Hall C alcove where it
passes the Compton and \$\text{M\o{}ller} polarimeters to check the
polarity of the beam. At this point, the beam has entered the hall where
it will travel to the scattering chamber and, any beam not incidented
off the target, will end its journey in the beam dump.

\begin{Mfigure}{Hall C Arc to Beamline}
  \centering
  \includegraphics[width=0.75\linewidth]{hallc_arc.pdf}
  \caption{The Hall C arc which steers the beam to the beamline. Electron synchrotron radiation loss is shown with the yellow arrows.}
  \label{fig:2-2_hallc_arc}
\end{Mfigure}



Along this path, there are several beam diagnostic components that track
and monitor various aspects of the beam. The harps, Beam Position
Monitors (BPMs), and Beam Current Monitors (BCMs) are the primary
components used for diagnostics.

\begin{Mfigure}{Hall C Beamline}
  \centering
  \includegraphics[width=0.75\linewidth]{beamline.pdf}
  \caption{Hall C beamline from entrance of hall to the target scattering chamber. [\cite{sta_jeerson_2019}]}
  \label{fig:2-2_beamline}
\end{Mfigure}



\hypertarget{beam-energy-measurement}{%
\subsection{Beam Energy Measurement}\label{beam-energy-measurement}}

The beam energy is determined by measuring the deflection of the
electron beam when it traverses through a known magnetic field in the
hall arc (see figure \ref{fig:2-2_hallc_arc}). In essence, the beam
energy is found by using the hall arc as a spectrometer
\cite{yan_beam_1993}. Using the basic description of a magnetic force
acting on an electron

\begin{equation} 
|\vec{F_{B}}|=e|\vec{v_e}\times\vec{B}|=ev_{e}B_{\perp}=\frac{\gamma m_{e}v^2_e}{r_c}
\label{eq:magnetic_force} 
\end{equation}

where \(e\) is the elementary charge , \(v_e\) is the electron velocity,
\(B_{\perp}\) is the magnetic field perpendicular to the velocity,
\(r_c\) is the radius of curvature, and \(\gamma\equiv(1-v^2_e/c^2)\).
Then using \(p_e=\gamma m_e v_e\) and \(r_c=dl/d\theta\), eq.
\ref{eq:magnetic_force} can be rewritten in terms of the electron
momentum

\begin{equation} 
p_e=\frac{e}{\theta_{arc}}\int B dl
\label{eq:electron_momentum} 
\end{equation}

where \(\theta_{arc}\) is the arc bend angle and \(dl\) is the
infintesimal arc length. Previous survey determined that the
\(\theta_{arc}\) was 34.3\(\degree\) and \(\int B dl\) is found by
mapping the magnetic fields of the arc dipoles at their corresponding
currents. The beam position and profile are measured superharps (see the
following sections) located at the entrance, middle, and exit of the
hall arc \cite{yan_superharp_1995}. Together there is an achievable
precision of \(\frac{\delta p}{p}\approx 5\times 10^{-4}\).

\hypertarget{beam-position-monitors-bpms}{%
\subsection{Beam Position Monitors
(BPMs)}\label{beam-position-monitors-bpms}}

The beam position and direction on the target is determined by three
BPMs, which can be found in on figure \ref{fig:2-2_beamline_components}.
Due to fringe fields of the SHMS magnets that arise from small forward
angle configurations, there are also two large diameter BPMs (known as
Big BPMs, see figure \ref{fig:2-2_beamline_bigbpms}). These BPMs are
cylindrical cavities consisting of a 4-wire antenna array which, to
minimizes synchrotron radiative damage, are rotated 45\(\degree\) with
respect to the horizontal and vertical axes. This array is made of thin
open ended wire striplines, used for RF signals requiring isolation from
surrounding circuitry, and tuned to \(f_0\). The beam induces an RF
signal in the antannae which is then either processed by the
Analog-to-Digital Converters (ADCs) or interpreted by the Experimental
Physics and Industrial Control System (EPICS).

\begin{Mfigure}{Hall C Beamline Components}
  \centering
  \includegraphics[width=0.75\linewidth]{beamline_components.pdf}
  \caption{The Hall C beamline components from entrance of hall to the target scattering chamber. The relevant distances from the scattering chamber are labeled.}
  \label{fig:2-2_beamline_components}
\end{Mfigure}



The BPMs can be read in by either of the two datastreams: EPICS or CODA.
The EPICS store the averaged position over 0.3 seconds while the
event-by-event information (i.e.~what is the processed by the ADC) is
stored by CODA. The raw beam postions from EPICS were used for this
experiment because of their simplicity
\footnote{The BPMS could have just as well been extracted from CODA, but this requires a separate calculation to convert the raw ADC to raw beam position values.}.
Once the raw beam values are obtained, the Hall C analysis software
(i.e.~calibrates relative to absolute beam position. The ratio of the
differences technique is used to determine the relative position of the
beam within 100 microns for currents above one \(\mu\)A. The superharps
(see next section) are calibrated with the BPMs to find the absolute
position.

\begin{Mfigure}{Hall C Beamline Big BPMS}
  \centering
  \includegraphics[width=0.75\linewidth]{beamline_bigbpms.pdf}
  \caption{Schematic of the Hall C beamline downstream, showing the Big BPMs (IPM3H08, IPM3H09) near the beam dump. [\cite{sta_jeerson_2019}]}
  \label{fig:2-2_beamline_bigbpms}
\end{Mfigure}



\hypertarget{harps}{%
\subsection{Harps}\label{harps}}

In order to obtain high precision measurements of the beam's profile and
position, the superharps are used. There are several superharps
throughout the beamline. There are two at the beginning and end of the
hall arc (see figure \ref{fig:2-2_hallc_arc}), one that is used by the
polarimeters (see figure \ref{fig:2-2_hallc_arc}), and the last two are
located just before the target (see figure
\ref{fig:2-2_beamline_components}).

Harps use a rotary encoder to determine absolute postion and three wires
connected to a fork to determine the profile. The superharp is attached
to a step motor where the linear movement is translated into rotary
motion. As the step motor moves, the encoder generates pulses equal to
the number of steps taken. This allows a relation between steps and beam
position which, along with the BPM information, is part of the
calibrations used in determining the absolute position. The step motor
movement also causes the wires the pass through the beam, where a signal
is generated in the form of a current produced by secondary electron
emission. This signal is amplified then sent to an ADC where the beam
profile can be obtained by fitting the spectrum. Since the harp moves
through the unrastered (see next section) beam, these ``harp scans'' are
done as an initial procedure when beam first becomes available.

\hypertarget{raster}{%
\subsection{Raster}\label{raster}}

The beam intensity can be so extreme that, after prolonged exposure,
there can be localized heating or damage to the target chamber or, even,
detectors. To prevent such damages, a raster is used to spread the beam
over a larger area, which, in turn, distributes the beam's power more
uniformly over the target. A raster applies small, controlled
oscillations, in both the horizontal and vertical directions, causing
the beam to scan back and forth or in a circular pattern, depending on
the desired beam profile. These oscillations are applied to the beam
steering elements, such as the magnets.

\begin{Mfigure}{XY Raster Plot}
  \centering
  \includegraphics[width=0.75\linewidth]{raster.pdf}
  \caption{Fast raster XY plot of run 6881 during 4.9 GeV. The uniform beam dispersion is evident by the teal homogeneity which corresponds to ~15 events per bin.}
  \label{fig:2-2_raster}
\end{Mfigure}



Hall C has three types of rasters available: M\(\o{}\)ller, fast, and
slow. The M\(\o{}\)ller and slow rasters are used for special
circumstances. The fast raster is primarily used. It works by driving AC
current from a 250 W audio amplifier to horizontal and vertical air-core
magnets, creating a rectangular beam dispersion. A 2x2 m\(m^2\) fast
raster was used throughout the experiment.

\hypertarget{beam-current-monitors-bcms}{%
\subsection{Beam Current Monitors
(BCMs)}\label{beam-current-monitors-bcms}}

Beam current is measured using BCMs, which, in Hall C, take two primary
forms: Unser monitor or RF cavity. An Unser monitor, also known as a
Parametric Current Transformer (PCT), is a toroidial transformer that is
designed to be non-destructive and acts as the absolute reference frame
\cite{unser_parametric_1992}. The circular magnetic field of the toroid
magnetizes strips of permeable matrial as the beam passes through it,
this sends a current, proportional to the beam current, to a
compensating coil thus canceling out the field of the beam. RF cavities,
on the other-hand, are cylindrical, stainless steal cavities that work
off the basic ideas of a waveguide. When the beam passes through them at
their resonant frequency (i.e.~\(f_0\)), they are excited and an antenna
inside couples this power to a heliax cable to eventually be processed.

An Unser monitor and two RF cavities (BCM1 and BCM2) wrapped in thermal
blankets, for temperature stabilization, make up the primary current
monitoring system and are located upstream from the target. There is
also an additional three RF cavities (BCM4A, BCM4B, and BCM4C) enclosed
in a thermally stabilized box that are aviable if the beamline
configuration allows. These are located further upstream than the Unser
or BCM1, just inside the hall entrance. Finally, there is one last BCM
(BCM17) located upstream from the Compton polarimeter on the 3C17
girder. In addition to these BCMs, that are part of the Hall C current
measurement system, there is another BCM immediately upstrem of the
target which is used to monitor beam loss and is primarily there for
machine protection.

\begin{Mfigure}{BCM1/2 and Unser}
  \centering
  \includegraphics[width=0.75\linewidth]{bcm_unser.pdf}
  \captionsetup{width=0.95\textwidth} % Adjust the caption width here
  \caption{Drawing showing BCM1 and BCM2 on either side of the Unser monitor. A Unser wire calibration can be performed by passing a wire (labeled calibration wire), with an accurately known current, through the Unser monitor to measure the gain of the electronic chain.}
  \label{fig:2-2_bcm_unser}
\end{Mfigure}



The signal for the Unser monitor drifts significantly over the course of
a few minutes, so it cannot be used as a continuous current monitor.
Still, being the absolute reference frame means that the Unser can be
used to calibrate the cavities which are used for continous current
monitoring. In order to correct for this drift, a ``Unser wire
calibration'' is performed, wherein, a wire, with a known current, is
run through the Unser and is used to measure the gain of the electronic
chain (see figure \ref{fig:2-2_bcm_unser}). This gain is then
calibrated, which corrects for the drift in the absolute frame. The
combination of the Unser monitor with BCM1 and BCM2 allows a beam
current with an absolute accuracy of about 1 \cite{denard_high_2001}.

\hypertarget{Section-4.4}{%
\section{Target}\label{Section-4.4}}

At the end of the beamline, the electron beam has finally reached the
target chamber. The target chamber was designed to isolate the beam line
vacuum from the rotating spectrometers. The chamber houses a loop of
cryogenic targets (see figure \ref{fig:2-3_target_loop}), such as liquid
hydrogen, as well as a variety of solid targets on the ladder, such as
Aluminum. All available targets during the E12-09-011 experiment can be
seen in table \ref{tab:2-3_target_loop}.

\begin{Mfigure}{Target Loop}
  \centering
  \includegraphics[width=0.75\linewidth]{target_loop.pdf}
  \caption{ CAD view of the cryogenic and solid target ladders \cite{sta_jeerson_2019}. There are three cyrogenic loops above the solid target ladder.}
  \label{fig:2-3_target_loop}
\end{Mfigure}



\begin{Mtable}{Target Loop}
  \centering
  \begin{tabular}{|c|c|c|c|}
    \hline
    \textbf{Loop Number} & \textbf{Target} & \textbf{Length} & \textbf{Current Limits} \\
    \hline
    2 & L$H_2$ & 10 cm & ???? \\
    - & Dummy & 10 cm  & ???? \\
    - & Optics 1 & 20 cm  & ???? \\
    - & Carbon Hole & ???  & ???? \\
    - & Carbon 1.5\% & ???  & ???? \\
    \hline
    \end{tabular}
  \caption{Break down of all targets available during the running period. The three cryogenic targets are labeled alongside their corresponding loops. The target length is also provided for each type.}
  \label{tab:2-3_target_loop}
\end{Mtable}



This experiment required a proton target so the primary one used was
liquid hydrogen (\(LH_2\)). The \(LH_2\) was contained in loop 2, while
the liquid Deuterium (\(LD_2\)) was in loop 3 and loop 3 was empty, only
containing \(^{2}\mathrm{He}\) gas to keep vacuum pressure. \(LH_2\) is
kept at a temperature of 19 \(\pm 0.1\) K(\textasciitilde25 psia) and
density of 0.07231 g/\(cm^3\) \cite{smith_g_hall_2016}. There needs to
be special care taken with all cryogenic targets as they have a strict
freezing and boiling points, therefore the temperatures need to be
closely monitored by the target operators to avoid disaster. For
\(LH_2\), the freezing and boiling point, respectively, are 13.8 K and
22.1 K.

The \(LH_2\) target may have been the main target used but it was not
the only one used. The 10 cm Aluminum Dummy target was also used
extensively. This solid target consists of aluminum foils mounted in
separate frame that correspond to the cryogenic entrance and exit
windows. This setup is purposeful as the 10 cm Aluminum Dummy target is
used for ``dummy subtractions'' which are the removal of background
events in the data associated with the aluminum frames that house the
cryogenic targets.

The the Carbon 1.5\% target was also used. Because these targets are
made of carbon, they withstand very high currents, which makes them
excellent for sieve and luminosity studies. These studies examine optics
and efficiencies, respectively, and will be discussed in more detail in
following sections and chapters.

Beyond these two targets used for data, there are also the Carbon Hole,
Optics-1 targets which are used for beam centering and spectrometer
optics, respectively. The Carbon Hole is a thin carbon foil that has
been cut so that there is a central 2 mm diameter hole. Similar to
figure \ref{fig:2-2_raster}, the rastered beam can be used to adjust the
beam to a more central position. With a Carbon Hole target, a rastered
XY plot will now have a hole showing the beam's central position. Shift
workers communicate adjustments to MCC so that it can be steered to its
proper position. The Optics-1 target is made of carbon foils located at
5 cm in front the entrance and 5 cm behind the exit cryogenic target
windows. These are used for spectrometer optics optimization studies,
which will be discussed in a following section.

\hypertarget{Section-4.5}{%
\section{Spectrometers}\label{Section-4.5}}

The distinctive feature of Hall C lies in its possession of two high
luminosity (\(10^{39}\text{cm}^{-2}\text{s}^{-1}\)) magnetic
spectrometers, namely the High Momentum Spectrometer (HMS) and the Super
High Momentum Spectrometer (SHMS). These cutting-edge instruments
empower physicists to conduct unparalleled precision in cross-section
experiments. Each spectrometer rests on rotatable support structures
that can move along rails, all while keeping the optical elements and
detectors aligned relative to the target.

\begin{Mtable}{Spectrometer Specifications}
  \centering
  \begin{tabular}{|c|c|c|c|}
    \hline
    \textbf{Parameter} & \textbf{HMS} & \textbf{SHMS}\\
    \hline
    Central Momentum & 0.4 - 7.4 GeV/c & 2 - 11 GeV/c \\
    Momentum Acceptance & $\pm$10\% & -10\% - +22\% \\
    Momentum Resolution & 0.1\% - 0.15\% & 0.03\% - 0.08\% \\
    Scattering Angle Range & 10.5$\degree$ - 90$\degree$ & 5.5$\degree$ - 40$\degree$ \\
    \hline
    Horizontal Angle Acceptance & $\pm$32 mrad & $\pm$18 mrad \\
    Horizontal Angle Resolution & 0.8 mrad & 0.5 - 1.2 mrad \\
    Vertical Angle Acceptance & $\pm$85 mrad & $\pm$50 mrad \\
    Vertical Angle Resolution & 1.0 mrad & 0.3 - 1.1 mrad \\
    Solid Angle Acceptance & 8.1 msr & >4 msr \\
    \hline
    Maximum Event Rate & 2000 Hz?? & 10,000 Hz?? \\
    e/h Discrimination & >1000:1 at 98\% efficiency?? & >1000:1 at 98\% efficiency?? \\
    $\pi$/K Discrimination & 100:1 at 95\% efficiency?? & 100:1 at 95\% efficiency?? \\
    \hline
  \end{tabular}
  \caption{Break down of the HMS and SHMS specifications and capablilities..}  
  \label{tab:2-4_spectrometer}
\end{Mtable}



A breakdown of the spectrometer specifications and capablilities are
given in table \ref{tab:2-4_spectrometer}. The sucess of the HMS at
hadron detection during the 6 GeV era inspired the SHMS. Because of
this, the two spectrometers are very similar in design. Inside their
heavily shielded detector hut, both consist of a series of dipoles and
quadrupoles followed by a lineup of particle detectors. The dipoles and
quadrupoles, together known as the optical elements, are used to steer
the scattered particles to the detectors from the target chamber. These
steered particles are then tracked through drift chambers (DC) by
reconstructing their trajectories from the focal plane back to the
target. These tracked particles are then identified through a series of
hodoscopes, Cerenkov detectors and calorimeters.

\begin{Mfigure}{Spectrometer Setup}
  \centering
  \includegraphics[width=0.75\linewidth]{spec_setup.pdf}
  \caption{Overview of Hall C Spectrometer setup. [\cite{sta_jeerson_2019}]}
  \label{fig:2-4_spec_setup}
\end{Mfigure}



\hypertarget{high-momentum-spectrometer-hms}{%
\subsection{High Momentum Spectrometer
(HMS)}\label{high-momentum-spectrometer-hms}}

The HMS (labeled on the right in figure \ref{fig:2-4_spec_setup}) was
used to detect electrons in this experiment. There are four optical
components of the HMS consisting of three quadrupoles and one dipole
(seen in figure \ref{fig:2-4_hms_magnets}) arranged in a QQQD
configuration. At the entrance of the first quadrupole, is a slit system
containing two collimators (the large and pion collimators) and a sieve
slit that are used to define the angular acceptance and for spectrometer
optical studies, respectively.

\begin{Mfigure}{HMS Magnets}
  \centering
  \includegraphics[width=0.75\linewidth]{hms_magnets.pdf}
  \caption{Overview of HMS optical setup.}
  \label{fig:2-4_hms_magnets}
\end{Mfigure}



The collimators are rectangular blocks (90\% W and 10\% Cu/Ni)
consisting of an octagonal-shaped aperture that restrict the path of
particles entering the spectrometers. By carefully selecting the size
and arrangement of the aperture, the collimator ensures that only
particles within a specific angular range, defined by the acceptance
angle, are allowed to pass through to the detectors. The outer size of
the collimators is 11.75'' vertical by 8.25'' horizontal and a thickness
of 2.5''.

The sieve slits are very similar to the collimators, but, instead of
having an octagonal-shaped aperture, there are many closely space holes
drilled into it. As particles pass through the sieve slit, their
trajectories are slightly altered due to holes they must pass through,
thus it can cause a known and controlled amount of deflection to the
particle trajectories. The deflection caused by the sieve slit can be
used to understand the optics of the spectrometer by comparing the
measured deflection to the known pattern of the sieve slit. The outer
size of the sieve slit is 10.00'' vertical by 8.25'' horizontal and a
thickness 1.25''.

\begin{Mtable}{HMS Slit System}
  \centering
  \begin{tabular}{|c|c|c|c|}
    \hline
    \textbf{Collimator} & \textbf{d$\Omega$ (msr)} & \textbf{Horizontal (mr)} & \textbf{Vertical (mr)} \\
    \hline
    Large Collimator & 6.74 & $\pm$27.5 & $\pm$70.0 \\
    Pion Collimator & 4.03 & $\pm$21.3 & $\pm$54.0 \\
    \hline
  \end{tabular}
  \caption{Breakdown of HMS slit system's apertures.}
  \label{tab:2-4_hms_slit}
\end{Mtable}



The quadrupoles and dipole are superconducting magnets that are used to
optimize the magnetic field strength and configuration to achieve
precise and accurate particle tracking within the spectrometer. The
quadrupole provides point-to-point focusing while the dipole bends the
scattered particles in the dispersive direction and determines the
central momentum of the spectrometer.

\hypertarget{super-high-momentum-spectrometer-shms}{%
\subsection{Super High Momentum Spectrometer
(SHMS)}\label{super-high-momentum-spectrometer-shms}}

The SHMS (labeled on the left in figure \ref{fig:2-4_spec_setup}) was
used to detect kaons in this experiment. There are five optical
components of the HMS consisting of three quadrupoles, one dipole, and a
horizontal bender (seen in figure \ref{fig:2-4_shms_magnets}) arranged
in a HBQQQD configuration. Similar to the HMS, there is a slit system
before the first quadrupole containing a collimator and two sieve slits.
This slit system has the same use as the HMS.

\begin{Mfigure}{SHMS Magnets}
  \centering
  \includegraphics[width=0.75\linewidth]{shms_magnets.pdf}
  \caption{Overview of SHMS optical setup.}
  \label{fig:2-4_shms_magnets}
\end{Mfigure}



The collimator is a rectangular block (90\% W, 6\% Ni and 4\% Cu)
consisting of an octagonal-shaped aperture. The outer size of the
collimators is 6.693'' vertical by 9.843'' horizontal and a thickness of
2.5''. The first sieve slit has 11 columns of holes, centered at the 6th
column. The second sieve slit, the shifted sieve slit, has 10 columns
and is offset by half a column from the center. This shifted sieve slit
is in place to study the optics of the horizontal bender, which lies
just before the slit system.

\begin{Mtable}{SHMS Slit System}
  \centering
  \begin{tabular}{|c|c|c|c|}
    \hline
    \textbf{Collimator} & \textbf{d$\Omega$ (msr)} & \textbf{Horizontal (mr)} & \textbf{Vertical (mr)} \\
    \hline
     Collimator & 4.00 & $\pm$24.0 & $\pm$40.0 \\
    \hline
  \end{tabular}
  \caption{Breakdown of SHMS slit system's apertures.}
  \label{tab:2-4_shms_slit}
\end{Mtable}



Like the HMS, the quadrupoles and dipole are superconducting magnets
that are used to optimize the magnetic field strength and configuration
to achieve precise and accurate particle tracking within the
spectrometer. At angles below \textasciitilde12\(\degree\), fringe
fields from the SHMS HB, Q1, and Q2 magnets can deflect the beam which
may result in it missing the beam dump. The Horizontal Bender (HB) is
used to steer the scattered particles by an additional 3\(\degree\) away
from the beamline in order to prevent this.

\hypertarget{Section-4.6}{%
\section{Detectors}\label{Section-4.6}}

The HMS and SHMS consist of a very similar set of particle detectors
that are aligned in almost exactly the same order. This was a purposeful
choice based off the success of previous Hall C electroproduction
experiments (see Ref. \cite{horn_determination_2006},
\cite{blok_charged_2008}) that used the HMS and the Short Orbit
Spectrometer (SOS), which the SHMS replaced. The basic design consists
of a pair of drift chambers (DC) for track reconstruction, two pairs of
hodoscopes for triggering and time-of-flight (TOF) measurements, and a
combination of Cerenkov detectors and calorimeters for particle
identification (PID).

\begin{Mfigure}{HMS Detector Stack}
  \centering
  \includegraphics[width=0.75\linewidth]{hms_detectors.pdf}
  \caption{Overview of HMS Detector Stack.}
  \label{fig:2-4_hms_detectors}
\end{Mfigure}



The HMS detector stack can be seen in figure
\ref{fig:2-4_hms_detectors}. As a scattered electron leaves the HMS
dipole exit window, it passes through the vacuum vessel which is
low-pressure environment used to removing air and other gases from the
vicinity of the detectors. This is essential because the presence of gas
molecules can cause scattering and interactions with the particles being
detected, potentially affecting the accuracy of the measurements. Once
the electron exits the vacuum vessel, it passes through the pair of
drift chambers (DC1 and DC2). In the E12-09-011 experiment, the HMS
aerogel detector was not installed because only electrons were detected
in this spectrometer and, as well be explained in the next sections, the
aerogel is used for \(p\)/\(\pi\)/\(K\) separation. Instead, the
electron will next pass through the first pair of hodoscopes (1X and
1Y). The electron will then pass through the HMS Cerenkov, then the
second pair of hodoscopes (2X and 2Y). Finally, the electron ends its
journey by going through the preshower and shower counters that make up
the HMS calorimeter.

\begin{Mfigure}{SHMS Detector Stack}
  \centering
  \includegraphics[width=0.75\linewidth]{shms_detectors.pdf}
  \caption{Overview of SHMS Detector Stack.}
  \label{fig:2-4_shms_detectors}
\end{Mfigure}



The SHMS detector stack can be seen in figure
\ref{fig:2-4_shms_detectors}. As a scattered kaon leaves the SHMS dipole
exit window, it passes through its own vacuum vessel which is extended
using a vacuum extension pipe. This is because the Noble Gas Cerenkov
(NGC) was not installed during this experiment, thus the extension pipe
is necessary. Like the electron in the HMS, the kaon will first travel
through a pair of drift chambers and the first pair of hodoscopes. The
kaon will then pass through the SHMS aerogel Cerenkov detector, followed
by the SHMS Heavy Gas Cerenkov (HGC) detector. Finally, the kaon will go
through the second pair of hodoscopes before ending in the preshower and
shower counters of the SHMS calorimeter.

\hypertarget{drift-chambers}{%
\subsection{Drift Chambers}\label{drift-chambers}}

Drift chambers measure the horizontal and vertical angles and positions
of the scattered particles. The charge of the particles induces
ionization in the gas of the chamber (50:50 argon/ethane) which produces
free electrons that are caught by sense wires. The HMS drift chambers
were upgraded in 2017 to match the design of the SHMS, which itself is
based off the previous chambers used in the Hall C program
\cite{pandey_status_2017} \cite{tang_hall_2017}
\cite{christy_hall_2016}. The basic design consists of two cathode
windows, eight cathode planes, six anode (wire) planes, two aluminum
frames, and a middle plane with card carriers and readout electronics
(see figure \ref{fig:2-4_dc_view}).

There are six planes of wires in all the drift chamber pairs (see figure
\ref{fig:2-4_dc_view}). The wires are set such that a 180\(\degree\)
rotation of the unprimed wire planes (e.g.~U) about the z-axis result in
the primed wire planes (e.g.~U'), but slightly shifted to resolve the
left-right ambiguity in the case of multi-hits. The x/x' and y/y' planes
determine the dispersive (vertical) and non-dispersive (horizontal)
track positions, respectively. To improve tracking resolution, the u/u'
and v/v' planes are set \(\pm60\degree\) relative to x/x'.

\begin{Mfigure}{Drift Chamber View}
  \centering
  \includegraphics[width=0.75\linewidth]{dc_view.pdf}
  \caption{Basic design and components of DC1 (left) and DC2 (right).}
  \label{fig:2-4_dc_view}
\end{Mfigure}



The two drift chambers (DC1 and DC2) are separated by the focal plane
which defines the focus point of the spectrometer optics. The focus
point is where the scattered particle should be when it is moving equal
to the central momentum. By knowing each component of the angles and
positions for both drift chambers, along with the location of the focal
plane, the charged particle's momenta and trajectories can be
calculated.

\hypertarget{hodoscopes}{%
\subsection{Hodoscopes}\label{hodoscopes}}

Two pairs of hodoscopes are used in each spectrometer. These use a very
basic design where each pair consists of two planes, perpendicular to
each other, and the planes consist of a series of detector paddles made
of long narrow strips of scintillator material (either plastic or
quartz) with photomultiplier tubes (PMTs) attached to both ends (see
figure \ref{fig:2-4_hodo_view}). In order to eliminate gaps between
elements, the scintillator paddles are arranged such that they overlap.
Plastic scintillators are used in the HMS and the first three planes of
the SHMS. The last plane in the SHMS uses a quartz scintillator. Since
the SHMS sees very high rates, the quartz scintillator improves the
already fast timing of the plastic scintillators to optimize the
hodoscope tracking efficiency.

\begin{Mfigure}{Hodoscope View}
  \centering
  \includegraphics[width=0.75\linewidth]{hodo_view.pdf}
  \caption{Basic design of a hodoscope pair.}
  \label{fig:2-4_hodo_view}
\end{Mfigure}



The hodoscopes are mainly used as triggers for particle events, although
they can also be used in TOF PID for lower momenta. When the scattered
particle passes through a hodoscope pair, the particle's charge ionizes
the scintillator material which excites its electrons. These electrons
fall back to their ground state and emit photons which bounce off of
reflective material that is wrapped around the scintillator. These
photons continue to propagate through the scintillator, via total
internal reflection, until they hit one of the two PMTs. The reflective
material is a layer of aluminum foil and multiple layers of Tedlar (HMS)
or electrical tape (SHMS). The quartz scintillator is slightly different
as it uses Cerenkov radiation (more on this in the next section) to
detect events.

\hypertarget{cerenkov-detectors}{%
\subsection{Cerenkov Detectors}\label{cerenkov-detectors}}

There are two types of Cerenkov detectors used in the spectrometers.
Both spectrometers use a Heavy Gas Cerenkov (HGC) detector and the SHMS
also uses an aerogel Cerenkov detector. The basic principle behind
Cerenkov detectors is the use of the Cerenkov effect, in which particles
passing through a medium travel faster than light in that medium. This
creates an effect analogous to a ``sonic boom,'' where a conic wave, of
light rather than sound, trails the particle. The angle that the light
is emitted is described by

\begin{equation} 
cos(\theta_c) = \frac{1}{n\beta}
\label{eq:theta_cer} 
\end{equation}

where \(n\) is the index of refraction of the medium and \(\beta=v/c\),
where \(v\) is the velocity of the particle in a vacuum. Since Cerenkov
light can only be produced if the particle travels faster than light in
the medium, \(\theta_c\) \textless{} \(\pi/2\) and we are left with
\(\beta > 1/n\). The velocity can be related to the momentum with
\(\beta=p/\sqrt{m^2+p^2}\) such that

\begin{equation} 
n >\frac{\sqrt{m^2+p^2}}{p}
\label{eq:index_cer} 
\end{equation}

and thus because certain index of refractions will produce light at
certain particle momenta, we can determine the PID from these Cerenkov
detectors.

\hypertarget{heavy-gas-cerenkov-detectors}{%
\subsubsection{Heavy Gas Cerenkov
Detectors}\label{heavy-gas-cerenkov-detectors}}

\begin{Mfigure}{SHMS HGC View}
  \centering
  \includegraphics[width=0.75\linewidth]{shms_hgc.pdf}
  \caption{View of the SHMS Heavy Gas Cerenkov's four mirrors and PMTs.}
  \label{fig:2-4_shms_hgc}
\end{Mfigure}



The HGC used in each spectrometer is a large cylindrical tank filled
with a gas (i.e.~the medium). The Cerenkov light is then reflected to
PMTs using mirrors. The HMS Cerenkov is 1.5 meters long and uses two
spherical mirrors that focus the light to two PMTs
\cite{noauthor_threshold_1995}. The gas of the HMS Cerenkov can be
adjusted to discriminate between either \(e\)/\(\pi\) (\(C_4F_{10}\) or
\(N_2\)) or \(\pi\)/\(p\) (Freon-12). The SHMS Cerenkov is 1.3 meters
long and uses four spherical mirrors that focus the light to four PMTs
\cite{li_heavy_2012}. The gas of the SHMS Cerenkov can be adjusted to
discriminate between either \(e\)/\(\pi\) or \(\pi\)/\(K\). The gases
used (\(C_4F_{10}\) or \(C_4F_8O\)) are set for different particle
separation by adjusting the pressure of the gas.

\hypertarget{shms-aerogel-cerenkov-detector}{%
\subsubsection{SHMS Aerogel Cerenkov
Detector}\label{shms-aerogel-cerenkov-detector}}

\begin{Mfigure}{SHMS Aerogel View}
  \centering
  \includegraphics[width=0.75\linewidth]{shms_aero.pdf}
  \caption{View of the SHMS aerogel Cerenkov.}
  \label{fig:2-4_shms_aero}
\end{Mfigure}



The aerogel Cerenkov detector used by the SHMS has two main components:
a tray to hold the aerogel material and a light diffusion box with PMTs,
which are both covered with a diffuse reflector material
\cite{horn_aerogel_2017}. In order to separate kaons at such a high
momentum (2.6 to 7.2 GeV/c), a refractive index between gases and
liquids is required. Aerogel is one of the few materials with such a
property. Aerogel is an extremely low density, near translucent
material. It is composed of a gel-like structure in which the liquid
component has been replaced with gas, resulting in a solid material that
is mostly composed of air. The Cerenkov effect is used in just the same
way as the HGC, only instead of mirrors, a diffuse reflector material is
used to reflect the Cerenkov light to the PMTs.

\begin{Mtable}{SHMS Aerogel Threshold Momenta}
  \centering
  \begin{tabular}{|c|c|c|c|c|}
    \hline
    \textbf{Particle} & \textbf{$P_{Th}$} & \textbf{$P_{Th}$} & \textbf{$P_{Th}$} & \textbf{$P_{Th}$} \\
    \hline
    $\mu$ & 0.428 & 0.526 & 0.608 & 0.711 \\
    $\pi$ & 0.565 & 0.692 & 0.803 & 0.935 \\
    $K$ & 2.000 & 2.453 & 2.840 & 3.315 \\
    $p$ & 3.802 & 4.667 & 5.379 & 6.307 \\
    \hline
  \end{tabular}
  \caption{Threshold momenta ($P_{Th}$ in GeV/c) for a variety of charged particles and the corresponding refractive indices.}
  \label{tab:2-4_aero_threshold}
\end{Mtable}



Four identical trays for aerogel of nominal refractive indices of 1.030,
1.020, 1.015, and 1.011 (also names SP-30, SP-20, SP-15, SP-11,
respectively) were used over the course of the experiment. The active
area for SP-11 was 60 cm width by 90 cm height and the rest of the
nominal refractive indices (SP-30, SP-20, SP-15) had an active area of
100 cm width by 110 cm height. A comparison of threshold momenta and
their corresponding nominal refractive indices can be seen in table
\ref{tab:2-4_aero_threshold}. The SP-30 and SP-20 aerogel trays have
their inner surfaces coated with a 0.45 μm thick Millipore paper
Membrane GSWP-0010 (Millipore), which has a reflectivity of
approximately 96\%. In the case of the SP-15 and SP-11 trays with lower
refractive indices, a 1 mm thick Gore diffusive reflector material
(DRP-1.0-12x30-PSA) with a reflectivity of about 99\% was used to
optimize light collection.

\input{figures/texs/fig:2-4_aero_tray.tex}

In order to exchange trays, the Hall C technical staff with experts'
assistance had to be brought in to assure a safe removal and
installation. Once all HV is turned off, the roof of the SHMS hut was
removed to make enough room for the tray to be removed via crane. The
crane holds the support structure for the tray while all bolts that
connect the tray with diffusion box are loosened. The support structure
fixes the tray such that the position is as the SHMS normal operational
angle (\textasciitilde18\(\degree\)). Once the straps of the support
structure are tightened, the bolts can be removed and the tray is slowly
disconnected from the diffusion box. The diffusion box and tray are
shielded to keep the volume clean. The crane is then used to lift the
tray out of the hut and onto a pallet, facing up. The tray is guided by
hand onto the tray to prevent shifting of the tiles. The procedure is
repeated in reverse when installing the new tray. This process was
repeated six???? times without issue.

\hypertarget{lead-glass-calorimeters}{%
\subsection{Lead Glass Calorimeters}\label{lead-glass-calorimeters}}

Both spectrometers use calorimeters with similar designs. The main
difference is that the SHMS has both a shower and preshower. By means of
Cerenkov light detection, the showers capture all the electromagnetic
(EM) showers (via PMTs) that are produced by Bremmstrahlung radiation
and pair production processes. The Bremmstrahlung radiation is produced
when a scattered particle is suddenly slowed down by the calorimeter
radiator. These Bremmstrahlung photons decay to \(e^-e^+\) pairs
(i.e.~pair production) which in turn emit Bremmstrahlung radiation. This
creates the EM shower that is captured by the shower. The preshower,
which is positioned in front of the shower, serves a similar purpose
only it has a shorter radiation length, which allows for PID by
distinguishing the early EM showers of an electron versus the later EM
showers of a hadron.

\begin{Mfigure}{SHMS Calorimeter}
  \centering
  \includegraphics[width=0.75\linewidth]{shms_cal.pdf}
  \caption{View of the SHMS calorimeter which shows the preshow and shower, along with their corresponding lead glass blocks. [\cite{mkrtchyan_lead-glass_2013}]}
  \label{fig:2-4_shms_cal}
\end{Mfigure}



The HMS and SHMS calorimeters are made of several stacked layers of
thick lead blocks to ensure that nearly all the scattered particle's
incident radiation is captured \cite{mkrtchyan_lead-glass_2013}. This
block stack is tilted by a a few degrees (5\(\degree\) for the HMS and
2\(\degree\) for the SHMS) relative to the central ray of the
spectrometer so that the losses due to particles passing through the
gaps between blocks can be minimized. The HMS uses 52 TF-1 lead glass 10
cm thick blocks for the shower. It has a radiation length of
\textasciitilde14.6. The SHMS uses 28 TF-1 lead glass 10 cm thick blocks
for the preshower and 224 F-101 lead glass 50 cm thick blocks for the
shower. The preshower has a radiation length of 3.6 and the shower has a
radiation length of 18.

\hypertarget{Section-4.7}{%
\section{Trigger Logic and Data Acquisition}\label{Section-4.7}}

The hardware trigger system is one of the main components of the data
acquisition. It is used to filter real events from likely backgrounds by
reducing the high rates and electronic deadtime while keeping the
trigger efficiency high. Each detector's output signal, whether that be
from PMTs or drift chambers, are digitized with high speed ADCs and
time-to-digital converters (TDCs).

FADC250 with 16 channel ADC modules, running with a 4 ns period, are
used to digitize the input signals from the calorimeters, Cerenkov
detectors and hodoscopes. CAEN V1190A TDCs with 128 channels each are
used to digitized discriminated signals with 100 ps of resolution, which
are used by hodoscopes and drift chambers. Most of these ADC/TDCs are
read out by Read-Out Controllers (ROCs) crates located in the Hall C
Counting House Electronics Room. The exceptions being, the HMS/SHMS
drift chamber TDCs and SHMS shower ADCs are read out by ROCs in their
own respective detector huts.

\hypertarget{hms-trigger-setup}{%
\subsection{HMS Trigger Setup}\label{hms-trigger-setup}}

Blank Text

\hypertarget{shms-trigger-setup}{%
\subsection{SHMS Trigger Setup}\label{shms-trigger-setup}}

Blank Text

\hypertarget{electronic-dead-time-monitor-edtm}{%
\subsection{Electronic Dead Time Monitor
(EDTM)}\label{electronic-dead-time-monitor-edtm}}

Blank Text \markedchapter{Data Analysis}{Data Analysis}\label{Chapter-5}

\hypertarget{Section-5.1}{%
\section{Cross Section Definiton}\label{Section-5.1}}

\hypertarget{Section-5.2}{%
\section{LT Separation Procedure}\label{Section-5.2}}

\hypertarget{Section-5.3}{%
\section{Python Analysis Framework}\label{Section-5.3}}

\hypertarget{Section-5.4}{%
\section{Event Reconstruction}\label{Section-5.4}}

\hypertarget{Section-5.5}{%
\section{Particle Identification}\label{Section-5.5}}

\hypertarget{Section-5.6}{%
\section{Efficiency Corrections}\label{Section-5.6}}

\hypertarget{Section-5.7}{%
\section{Experimental Offsets}\label{Section-5.7}}

\markedchapter{Monte Carlo Simulations}{Monte Carlo Simulations}\label{Chapter-6}

\hypertarget{Section-6.1}{%
\section{Overview}\label{Section-6.1}}

\hypertarget{Section-6.2}{%
\section{Event Generation}\label{Section-6.2}}

\hypertarget{Section-6.3}{%
\section{Spectrometer Models}\label{Section-6.3}}

\hypertarget{Section-6.4}{%
\section{Material Interactions}\label{Section-6.4}}

\hypertarget{Section-6.5}{%
\section{Kaon Decay}\label{Section-6.5}}

\hypertarget{Section-6.6}{%
\section{Radiative Corrections}\label{Section-6.6}}

\hypertarget{Section-6.7}{%
\section{Elastic Scattering}\label{Section-6.7}}

\markedchapter{Experimental Cross Section Overview}{Experimental Cross Section Overview}\label{Chapter-7}

\hypertarget{Section-7.1}{%
\section{Monte Carlo Equivalent Yield}\label{Section-7.1}}

\hypertarget{Section-7.2}{%
\section{Cross Section Determination}\label{Section-7.2}}

\hypertarget{Section-7.3}{%
\section{Model Cross Section}\label{Section-7.3}}

\hypertarget{Section-7.4}{%
\section{Model and Data Comparison}\label{Section-7.4}}

\hypertarget{Section-7.5}{%
\section{Error Analysis}\label{Section-7.5}}

\markedchapter{KaonLT Results and Discussion}{KaonLT Results and Discussion}\label{Chapter-8}

\hypertarget{Section-8.1}{%
\section{Overview}\label{Section-8.1}}

\hypertarget{Section-8.2}{%
\section{Experimental Cross Sections}\label{Section-8.2}}

\hypertarget{Section-8.3}{%
\section{Unseparated Cross Sections}\label{Section-8.3}}

\hypertarget{Section-8.4}{%
\section{Separated Cross Sections}\label{Section-8.4}}

\hypertarget{Section-8.5}{%
\section{\texorpdfstring{Extraction of \(K^{+}\) Form
Factor}{Extraction of K\^{}\{+\} Form Factor}}\label{Section-8.5}}

\hypertarget{Section-8.6}{%
\section{Discussion}\label{Section-8.6}}

\markedchapter{Future Measurements at EIC}{Future Measurements at EIC}\label{Chapter-9}

\hypertarget{Section-9.1}{%
\section{Overview}\label{Section-9.1}}

\hypertarget{Section-9.2}{%
\section{Meson Structure Function}\label{Section-9.2}}

Theory

\hypertarget{Section-9.3}{%
\section{Experimental Considerations}\label{Section-9.3}}

\hypertarget{Section-9.4}{%
\section{Far Forward Detection and Simulations}\label{Section-9.4}}

MC+GEANT

\hypertarget{Section-9.5}{%
\section{Structure Function Projections}\label{Section-9.5}}

\hypertarget{Section-9.6}{%
\section{Discussion and Outlook}\label{Section-9.6}}

\markedchapter{Conclusion}{Conclusion}\label{Chapter-10}

\hypertarget{Section-10.1}{%
\section{Jlab, EIC, and Beyond}\label{Section-10.1}}

\end{document}
