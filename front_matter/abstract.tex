\chapter[Abstract]{\label{Abstract}}

\begin{center}
  {\large Determination of Pion and Kaon Structure Using the Sullivan Process at Moderate to Large Fractional Momentum\par}
  \vspace{0.5cm}
  {\large Richard L. Trotta, Ph.D.\par}
  {Director: Tanja Horn, Ph.D.\par}
\end{center}

\vspace{0.5cm}

Significant progress has been made in describing the true hadron structure in terms of Quantum ChromoDynamics (QCD) degrees of freedom. In particular, the Generalized Parton Distributions (GPDs) framework has developed three-dimensional imaging of the hadronic structure. Integrating these structure functions over the fractional momentum, the form factors are obtained which are integrally related to understanding the emergent hadron mass-generating mechanism (EHM) embedded in QCD and the mass-scale that characterizes the proton. The Sullivan process, in which one scatters electrons off the proton’s virtual meson cloud and thus creates a meson target, is used to probe this structure. Studies of kaon structure offer unique opportunities in this. The last two decades saw a dramatic improvement in the precision of charged pion form factor data, and new results have become available on the transition form factor. This experiment, KaonLT, measures longitudinal/transverse (L/T) separated kaon cross sections required to access form factors and GPDs. This experiment aims to confirm the exclusive kaon electroproduction mechanism, which is essential for determining the kaon form factor and investigating GPDs. The results obtained from this experiment have provided the highest $Q^2$ for L-T separated kaon electroproduction cross section, and it is the first time that a separated kaon cross-section measurement has been carried out above $W=2.2 \mathrm{GeV}$.

\thispagestyle{empty} % Hide page number for this page