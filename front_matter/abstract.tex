\chapter[Abstract]{Abstract}

\begin{center}
Determination of Pion and Kaon Structure Using the Sullivan Process at Moderate to Large Fractional Momentum

Richard L. Trotta, Ph.D.
\end{center}

There has been significant progress in the description of the true hadron structure in terms of Quantum ChromoDynamics (QCD) degrees of freedom. In particular, the framework of Generalized Parton Distributions (GPDs) has developed three-dimensional imaging of the hadronic structure [1]. Integrating these structure functions over the fractional momentum (denoted x), one can obtain the form factors which can be probed experimentally and provide the physical charge and magnetic distributions. These form factors are integrally related to our understanding of the emergent hadron mass-generating mechanism (EHM) embedded in QCD and the mass-scale that characterizes the proton and thus all observable matter. Studies of pion and kaon structure offer unique opportunities in this. The last two decades saw a dramatic improvement in the precision of charged pion form factor data, and new results have become available on the transition form factor. The kaon provides an interesting way to expand these studies, opening the possibility to access the production mechanism involving strangeness. My thesis experiment: KaonLT measures longitudinal/transverse (L/T) separated kaon cross sections that are required for accessing form factors and GPDs [2,3]. Such measurements require excellent control of systematic uncertainties (e.g., the acceptance) and can only be accomplished with the magnetic spectrometers in Hall C at Jefferson Lab. The purpose of my PhD studies specifically is to validate the exclusive kaon electroproduction mechanism, the prerequisite for extracting the kaon form factor and probing GPDs. My results will yield the highest $Q^2$ for L-T separated kaon electroproduction cross section, and the first separated kaon cross section measurement above W=2.2 GeV.
The understanding of the pion and kaon mass acquisition through EHM can also be probed through studies of Parton Distribution Functions (PDF). Similar to the measurement of the form factors one uses the Sullivan process in which one scatters electrons of the virtual pion/kaon cloud of the proton, and thus creates a pion/kaon target. This process is valid if the pole associated with the ground-state pion/kaon remains dominant and the structure of the related correlation evolves slowly and smoothly with virtuality. To check if these conditions are satisfied empirically, one can take data covering a range in the Mandelstam variable t, particularly low t, and compare it with phenomenological and theoretical expectations. The validation of the Sullivan process is synergistic between form factor and PDF measurements. My results from my KaonLT cross section analysis may provide further insight into planned PDF studies at the Electron-Ion Collider (EIC); a new \$2B high-luminosity facility capable of colliding high-energy electron beams with high-energy proton and ion beams. The EIC will be built at BNL in approximately 10 years in order to unlock the secrets of the “glue” that binds the building blocks of visible matter in the universe [4]. I made initial projections of kaon and pion structure function measurements at an EIC. These suggest a substantial gain in constraining the meson PDFs and significant improvement in valence quark, sea quark, and gluon extractions.


