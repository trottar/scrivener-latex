\markedchapter{Introduction}{Introduction}\label{Chapter-2}

\begin{quote}
\centering \emph{"To a man devoid of blinders, there is no finer sight than that of intelligence at grips with a reality that transcends it."}\\
\centering \emph{- Albert Camus, Myth of Sisyphus}
\end{quote}

\hfill

The meaning of life and existence has fascinated humanity for all of
written history. Starting with the foundations of mathematics and
philosophy by the ancient Greeks through the time of enlightenment, the
rationality driven by the materialistic world has shaped our current
society.~ \textless Describe the close relation of physics and
philosophy and how the two became so separated throughout the modern
era.\textgreater{}

Society in the 21st century is much different now. The modern era of
philosophy was~ coming to an end and a new era of post-modern
hyperreality started. (Dostoevsky said, ``if God is dead, then
everything is permitted'') Beginning~with Fredrick Nietzsche's famous
``God is dead'', the next 60 years was categorized years by a dichotomy
of profound scientific and unprecedented tragedy. Foundations of science
and society had been shattered and needed to be built anew. The theories
of general relativity and quantum mechanics rewrote the textbooks, just
as war unlike any ever seen were coming ahead. These two fronts collided
in the creation of nuclear weapons which provided countries with weapons
that hitherto been inconceivable.

The world changed forever in this moment. This was the moment where

\hypertarget{Section-2.1}{%
\section{Historical Context}\label{Section-2.1}}


