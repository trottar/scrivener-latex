\begin{figure}
  \centering
  \begin{minipage}[b]{0.48\linewidth}
    \includegraphics[width=\linewidth]{hgcer_hole_3233.pdf}
    \subcaption{Run 3233}
  \end{minipage}
  \hfill
  \begin{minipage}[b]{0.48\linewidth}
    \includegraphics[width=\linewidth]{hgcer_hole_4787.pdf}
    \subcaption{Run 4787}
  \end{minipage}
  
  \vspace{0.5cm}
  
  \begin{minipage}[b]{0.48\linewidth}
    \includegraphics[width=\linewidth]{hgcer_hole_6619.pdf}
    \subcaption{Run 6619}
  \end{minipage}
  \hfill
  \begin{minipage}[b]{0.48\linewidth}
    \includegraphics[width=\linewidth]{hgcer_hole_7095.pdf}
    \subcaption{Run 7095}
  \end{minipage}
  
  \caption{Comparison of the SHMS heavy gas cerenkov hole for runs 3233 (a), 4783 (b), 6619 (c), and 7095 (d). Run 3233 is the original HGCer configuration, run 4783 is after the first mirror adjustments and the optics replaced with an Acrylic ring, run 6619 is with the cones installed, and run 7095 is with the second alignment and with the cones installed. All runs are for electron PID (Calorimeter cut > 0.9) and with consistent calibrations. The NPE increased with each iteration, but neither alignment decreased the hole back to its original size.}
  \label{fig:3-3_hgcer_hole}
\end{figure}

