%%%%%%%%%% Example %%%%%%%%%%

%\begin{Mfigure}{Lorem Ipsum}
% \centering
% \includegraphics[width=\linewidth]{Figure-C1.pdf}
% \caption[\figuretitle]
{\textbf{(A) Lorem Ipsum.} Add as much text here as you want.}
%\label{fig:c1}
%\end{Mfigure}

\begin{Mfigure}{CUA Logo}
 \centering
 \includegraphics[width=0.5\linewidth]{CUA_logo.pdf}
\label{fig:titlepagelogo}
\end{Mfigure}

%%%%%%%%%%%%%%%%%%%%%%%%%%%%%%%%%%%%%%%%%%%%%%%%%%%%%%%%%%%%%%%%%%%%%%%%%%%%%%%%%%%%%%%%%%%%%%%%%%%%%%%%%%%%%%%%%%%%%%%%
%
% Chapter 2, section 1
%
%%%%%%%%%%%%%%%%%%%%%%%%%%%%%%%%%%%%%%%%%%%%%%%%%%%%%%%%%%%%%%%%%%%%%%%%%%%%%%%%%%%%%%%%%%%%%%%%%%%%%%%%%%%%%%%%%%%%%%%%

\begin{Mfigure}{Schematic of CEBAF 12 GeV Upgrade}
  \centering
  \includegraphics[width=0.75\linewidth]{cebaf_12gev.pdf}
  \caption{Schematic of CEBAF 12 GeV Upgrade.}
  \label{fig:2-1_cebaf12gev}
\end{Mfigure}

%%%%%%%%%%%%%%%%%%%%%%%%%%%%%%%%%%%%%%%%%%%%%%%%%%%%%%%%%%%%%%%%%%%%%%%%%%%%%%%%%%%%%%%%%%%%%%%%%%%%%%%%%%%%%%%%%%%%%%%%
%
% Chapter 2, section 2
%
%%%%%%%%%%%%%%%%%%%%%%%%%%%%%%%%%%%%%%%%%%%%%%%%%%%%%%%%%%%%%%%%%%%%%%%%%%%%%%%%%%%%%%%%%%%%%%%%%%%%%%%%%%%%%%%%%%%%%%%%

\begin{Mfigure}{Hall C Arc to Beamline}
  \centering
  \includegraphics[width=0.75\linewidth]{hallc_arc.pdf}
  \caption{The Hall C arc which steers the beam to the beamline. Electron synchrotron radiation loss is shown with the yellow arrows.}
  \label{fig:2-2_hallc_arc}
\end{Mfigure}

\begin{Mfigure}{Hall C Beamline}
  \centering
  \includegraphics[width=0.75\linewidth]{beamline.pdf}
  \caption{Hall C beamline from entrance of hall to the target scattering chamber.}
  \label{fig:2-2_beamline}
\end{Mfigure}

\begin{Mfigure}{Hall C Beamline Components}
  \centering
  \includegraphics[width=0.75\linewidth]{beamline_components.pdf}
  \caption{The Hall C beamline components from entrance of hall to the target scattering chamber. The relevant distances from the scattering chamber are labeled.}
  \label{fig:2-2_beamline_components}
\end{Mfigure}

\begin{Mfigure}{Hall C Beamline Big BPMS}
  \centering
  \includegraphics[width=0.75\linewidth]{beamline_bigbpms.pdf}
  \caption{Schematic of the Hall C beamline downstream, showing the Big BPMs (IPM3H08, IPM3H09) near the beam dump..}
  \label{fig:2-2_beamline_bigbpms}
\end{Mfigure}

\begin{Mfigure}{XY Raster Plot}
  \centering
  \includegraphics[width=0.75\linewidth]{raster.pdf}
  \caption{Fast raster XY plot of run 6881 during 4.9 GeV. The uniform beam dispersion is evident by the teal homogeneity which corresponds to ~15 events per bin.}
  \label{fig:2-2_raster}
\end{Mfigure}

\begin{Mfigure}{BCM1/2 and Unser}
  \centering
  \includegraphics[width=0.75\linewidth]{bcm_unser.pdf}
  \captionsetup{width=0.95\textwidth} % Adjust the caption width here
  \caption{Drawing showing BCM1 and BCM2 on either side of the Unser monitor. A Unser wire calibration can be performed by passing a wire (labeled calibration wire), with an accurately known current, through the Unser monitor to measure the gain of the electronic chain.}
  \label{fig:2-2_bcm_unser}
\end{Mfigure}

%%%%%%%%%%%%%%%%%%%%%%%%%%%%%%%%%%%%%%%%%%%%%%%%%%%%%%%%%%%%%%%%%%%%%%%%%%%%%%%%%%%%%%%%%%%%%%%%%%%%%%%%%%%%%%%%%%%%%%%%
%
% Chapter 2, section 3
%
%%%%%%%%%%%%%%%%%%%%%%%%%%%%%%%%%%%%%%%%%%%%%%%%%%%%%%%%%%%%%%%%%%%%%%%%%%%%%%%%%%%%%%%%%%%%%%%%%%%%%%%%%%%%%%%%%%%%%%%%

\begin{Mfigure}{Target Loop}
  \centering
  \includegraphics[width=0.75\linewidth]{target_loop.pdf}
  \caption{ CAD view of the cryogenic and solid target ladders. There are three cyrogenic loops above the solid target ladder.}
  \label{fig:2-3_target_loop}
\end{Mfigure}

\begin{Mtable}{Target Loop}
  \centering
  \begin{tabular}{|c|c|c|c|}
    \hline
    \textbf{Loop Number} & \textbf{Target} & \textbf{Length} & \textbf{Current Limits} \\
    \hline
    2 & L$H_2$ & 10 cm & ???? \\
    - & Dummy & 10 cm  & ???? \\
    - & Optics 1 & 20 cm  & ???? \\
    - & Carbon Hole & ???  & ???? \\
    - & Carbon 1.5\% & ???  & ???? \\
    \hline
    \end{tabular}
  \caption{Break down of all targets available during the running period. The three cryogenic targets are labeled alongside their corresponding loops. The target length is also provided for each type.}
  \label{tab:2-3_target_loop}
\end{Mtable}

\begin{Mtable}{LH2 Information}
  \centering
  \begin{tabular}{|c|c|c|c|}
    \hline
    \textbf{Target} & \textbf{Target Density} & \textbf{Target Length} & \textbf{Target Thickness} \\
    \hline
    $LH_2$ & 0.07231 g/$cm^3$ & 10 cm & mm \\
    \hline
    \end{tabular}
  \label{tab:2-3_lh2_properties}
\end{Mtable}

%%%%%%%%%%%%%%%%%%%%%%%%%%%%%%%%%%%%%%%%%%%%%%%%%%%%%%%%%%%%%%%%%%%%%%%%%%%%%%%%%%%%%%%%%%%%%%%%%%%%%%%%%%%%%%%%%%%%%%%%
%
% Chapter 2, section 4
%
%%%%%%%%%%%%%%%%%%%%%%%%%%%%%%%%%%%%%%%%%%%%%%%%%%%%%%%%%%%%%%%%%%%%%%%%%%%%%%%%%%%%%%%%%%%%%%%%%%%%%%%%%%%%%%%%%%%%%%%%

\begin{Mtable}{Spectrometer Specifications}
  \centering
  \begin{tabular}{|c|c|c|c|}
    \hline
    \textbf{Parameter} & \textbf{HMS} & \textbf{SHMS}\\
    \hline
    Central Momentum & 0.4 - 7.4 GeV/c & 2 - 11 GeV/c \\
    Momentum Acceptance & $\pm$10\% & -10\% - +22\% \\
    Momentum Resolution & 0.1\% - 0.15\% & 0.03\% - 0.08\% \\
    Scattering Angle Range & 10.5$\degree$ - 90$\degree$ & 5.5$\degree$ - 40$\degree$ \\
    \hline
    Horizontal Angle Acceptance & $\pm$32 mrad & $\pm$18 mrad \\
    Horizontal Angle Resolution & 0.8 mrad & 0.5 - 1.2 mrad \\
    Vertical Angle Acceptance & $\pm$85 mrad & $\pm$50 mrad \\
    Vertical Angle Resolution & 1.0 mrad & 0.3 - 1.1 mrad \\
    Solid Angle Acceptance & 8.1 msr & >4 msr \\
    \hline
    Maximum Event Rate & 2000 Hz?? & 10,000 Hz?? \\
    e/h Discrimination & >1000:1 at 98\% efficiency?? & >1000:1 at 98\% efficiency?? \\
    $\pi$/K Discrimination & 100:1 at 95\% efficiency?? & 100:1 at 95\% efficiency?? \\
    \hline
  \end{tabular}
  \caption{Break down of the HMS and SHMS specifications and capablilities..}  
  \label{tab:2-4_spectrometer}
\end{Mtable}

\begin{Mfigure}{Spectrometer Setup}
  \centering
  \includegraphics[width=0.75\linewidth]{spec_setup.pdf}
  \caption{Overview of Hall C Spectrometer setup.}
  \label{fig:2-4_spec_setup}
\end{Mfigure}

\begin{Mfigure}{HMS Magnets}
  \centering
  \includegraphics[width=0.75\linewidth]{hms_magnets.pdf}
  \caption{Overview of HMS optical setup.}
  \label{fig:2-4_hms_magnets}
\end{Mfigure}

\begin{Mtable}{HMS Slit System}
  \centering
  \begin{tabular}{|c|c|c|c|}
    \hline
    \textbf{Collimator} & \textbf{d$\Omega$ (msr)} & \textbf{Horizontal (mr)} & \textbf{Vertical (mr)} \\
    \hline
    Large Collimator & 6.74 & $\pm$27.5 & $\pm$70.0 \\
    Pion Collimator & 4.03 & $\pm$21.3 & $\pm$54.0 \\
    \hline
  \end{tabular}
  \caption{Breakdown of HMS slit system's apertures.}
  \label{tab:2-4_hms_slit}
\end{Mtable}

\begin{Mfigure}{SHMS Magnets}
  \centering
  \includegraphics[width=0.75\linewidth]{shms_magnets.pdf}
  \caption{Overview of SHMS optical setup.}
  \label{fig:2-4_shms_magnets}
\end{Mfigure}

\begin{Mtable}{SHMS Slit System}
  \centering
  \begin{tabular}{|c|c|c|c|}
    \hline
    \textbf{Collimator} & \textbf{d$\Omega$ (msr)} & \textbf{Horizontal (mr)} & \textbf{Vertical (mr)} \\
    \hline
     Collimator & 4.00 & $\pm$24.0 & $\pm$40.0 \\
    \hline
  \end{tabular}
  \caption{Breakdown of SHMS slit system's apertures.}
  \label{tab:2-4_shms_slit}
\end{Mtable}

\begin{Mfigure}{HMS Detector Stack}
  \centering
  \includegraphics[width=0.75\linewidth]{hms_detectors.pdf}
  \caption{Overview of HMS Detector Stack.}
  \label{fig:2-4_hms_detectors}
\end{Mfigure}

\begin{Mfigure}{SHMS Detector Stack}
  \centering
  \includegraphics[width=0.75\linewidth]{shms_detectors.pdf}
  \caption{Overview of SHMS Detector Stack.}
  \label{fig:2-4_shms_detectors}
\end{Mfigure}

\begin{Mfigure}{Drift Chamber View}
  \centering
  \includegraphics[width=0.75\linewidth]{dc_view.pdf}
  \caption{Basic design and components of DC1 (left) and DC2 (right).}
  \label{fig:2-4_dc_view}
\end{Mfigure}

\begin{Mfigure}{Hodoscope View}
  \centering
  \includegraphics[width=0.75\linewidth]{hodo_view.pdf}
  \caption{Basic design of a hodoscope pair.}
  \label{fig:2-4_hodo_view}
\end{Mfigure}

\begin{Mfigure}{SHMS HGC View}
  \centering
  \includegraphics[width=0.75\linewidth]{shms_hgc.pdf}
  \caption{View of the SHMS Heavy Gas Cerenkov's four mirrors and PMTs.}
  \label{fig:2-4_shms_hgc}
\end{Mfigure}

\begin{Mfigure}{SHMS Aerogel View}
  \centering
  \includegraphics[width=0.75\linewidth]{shms_aero.pdf}
  \caption{View of the SHMS aerogel Cerenkov.}
  \label{fig:2-4_shms_aero}
\end{Mfigure}

\begin{Mtable}{SHMS Aerogel Threshold Momenta}
  \centering
  \begin{tabular}{|c|c|c|c|c|}
    \hline
    \textbf{Particle} & \textbf{$P_{Th}$} & \textbf{$P_{Th}$} & \textbf{$P_{Th}$} & \textbf{$P_{Th}$} \\
    \hline
    $\mu$ & 0.428 & 0.526 & 0.608 & 0.711 \\
    $\pi$ & 0.565 & 0.692 & 0.803 & 0.935 \\
    $K$ & 2.000 & 2.453 & 2.840 & 3.315 \\
    $p$ & 3.802 & 4.667 & 5.379 & 6.307 \\
    \hline
  \end{tabular}
  \caption{Threshold momenta ($P_{Th}$ in GeV/c) for a variety of charged particles and the corresponding refractive indices.}
  \label{tab:2-4_aero_threshold}
\end{Mtable}

\begin{Mfigure}{SHMS Aerogel Tray Exchange}
  \centering
  \includegraphics[width=0.5\linewidth]{aero_tray.pdf}
  \caption{View of an aerogel tray being lifted out of the SHMS hut. The tray maintains its 18$\degree$ angle until it is slowly lowerd on the pallet.}
  \label{fig:2-4_aero_tray}
\end{Mfigure}

\begin{Mfigure}{SHMS Calorimeter}
  \centering
  \includegraphics[width=0.75\linewidth]{shms_cal.pdf}
  \caption{View of the SHMS calorimeter which shows the preshow and shower, along with their corresponding lead glass blocks.}
  \label{fig:2-4_shms_cal}
\end{Mfigure}

%%%%%%%%%%%%%%%%%%%%%%%%%%%%%%%%%%%%%%%%%%%%%%%%%%%%%%%%%%%%%%%%%%%%%%%%%%%%%%%%%%%%%%%%%%%%%%%%%%%%%%%%%%%%%%%%%%%%%%%%
%
% Chapter 2, section 5
%
%%%%%%%%%%%%%%%%%%%%%%%%%%%%%%%%%%%%%%%%%%%%%%%%%%%%%%%%%%%%%%%%%%%%%%%%%%%%%%%%%%%%%%%%%%%%%%%%%%%%%%%%%%%%%%%%%%%%%%%%

\begin{Mfigure}{SHMS Trigger and DAQ}
  \centering
  \includegraphics[width=0.75\linewidth]{shms_trigger.pdf}
  \caption{Overview of the SHMS trigger and DAQ system. The HMS shares a similar trigger and DAQ system.}
  \label{fig:2-5_shms_trigger}
\end{Mfigure}

\begin{Mtable}{Hall C Pre-triggers}
  \centering
  \begin{tabular}{|c|c|}
    \hline
    \textbf{Input Triggers} & \textbf{Pre-trigger} \\
    \hline    
    \text{pTRIG1} & \text{pHODO 3/4} \\
    \text{pTRIG2} & \text{pEL REAL} \\
    \text{pTRIG3} & \text{hEL REAL} \\
    \text{pTRIG4} & \text{hHODO 3/4} \\
    \text{pTRIG5} & \text{hEL REAL + pHODO 3/4} \\
    \text{pTRIG6} & \text{hEL REAL + pEL REAL} \\
    \hline
  \end{tabular}
  \caption{The input triggers of the TM (i.e. ROC 02) compared to their corresponding single-arm ptr-trigger combinations.}
  \label{tab:2-5_pretriggers}
\end{Mtable}

\begin{Mtable}{Hall C Pre-triggers}
  \centering
  \begin{tabular}{|c|c|}
    \hline    
    \text{reference time}  = \text{TDC Start, single-arm pre-trigger} - \text{TDC Stop, L1ACCP} \\
    \text{detector raw time}  = \text{TDC Start, detector signal} - \text{TDC Stop, L1ACCP} \\
    \text{detector time}  = \text{detector raw time} - \text{reference time} \\
    \hline    
  \end{tabular}
  \label{tab:2-5_ref_sub}
\end{Mtable}

\begin{Mtable}{Hall C Pre-triggers}
  \centering
  \begin{tabular}{|c|c|c|}
    \hline
    \textbf{Value} & \textbf{Prescale Factor} & \textbf{Rate Prescaled from 100 kHz} \\
    \hline    
    0   &        1 & 100000 \\
    1   &        2 & 50000 \\
    2   &        3 & 33333 \\
    3   &        5 & 20000 \\
    4   &        9 & 11111 \\
    5   &       17 & 5882 \\
    6   &       33 & 3030 \\
    7   &       65 & 1538 \\
    8   &      129 & 775 \\
    9   &      257 & 389 \\
    0   &      513 & 194 \\
    1   &     1025 & 97 \\
    2   &     2049 & 48 \\
    3   &     4097 & 24 \\
    4   &     8193 & 12 \\
    5   &    16385 & 6 \\
    6   &    32769 & 3 \\
    \hline
  \end{tabular}
  \caption{List of prescale values to their corresponding factor. The third column shows an example of a prescaled rate from 100 kHz.}
  \label{tab:2-5_prescale}
\end{Mtable}


%%%%%%%%%%%%%%%%%%%%%%%%%%%%%%%%%%%%%%%%%%%%%%%%%%%%%%%%%%%%%%%%%%%%%%%%%%%%%%%%%%%%%%%%%%%%%%%%%%%%%%%%%%%%%%%%%%%%%%%%
%
% Chapter 3, section 1
%
%%%%%%%%%%%%%%%%%%%%%%%%%%%%%%%%%%%%%%%%%%%%%%%%%%%%%%%%%%%%%%%%%%%%%%%%%%%%%%%%%%%%%%%%%%%%%%%%%%%%%%%%%%%%%%%%%%%%%%%%

\begin{Mfigure}{Stage 1 of Analysis Procedure}
  \centering
  \includegraphics[width=0.85\linewidth]{LT_Analysis_Workflow_1.pdf}
  \caption{First step of the analysis workfow.}
  \label{fig:3-1_LT_Analysis_Workflow_1}
\end{Mfigure}

\begin{Mfigure}{Stage 2 of Analysis Procedure}
  \centering
  \includegraphics[width=0.85\linewidth]{LT_Analysis_Workflow_2.pdf}
  \caption{Second step of the analysis workfow.} 
  \label{fig:3-1_LT_Analysis_Workflow_2}
\end{Mfigure}

\begin{Mfigure}{Stage 3 of Analysis Procedure}
  \centering
  \includegraphics[width=0.85\linewidth]{LT_Analysis_Workflow_3.pdf}
  \caption{Final step of the analysis workfow.}
  \label{fig:3-1_LT_Analysis_Workflow_3}
\end{Mfigure}

%%%%%%%%%%%%%%%%%%%%%%%%%%%%%%%%%%%%%%%%%%%%%%%%%%%%%%%%%%%%%%%%%%%%%%%%%%%%%%%%%%%%%%%%%%%%%%%%%%%%%%%%%%%%%%%%%%%%%%%%
%
% Chapter 3, section 3
%
%%%%%%%%%%%%%%%%%%%%%%%%%%%%%%%%%%%%%%%%%%%%%%%%%%%%%%%%%%%%%%%%%%%%%%%%%%%%%%%%%%%%%%%%%%%%%%%%%%%%%%%%%%%%%%%%%%%%%%%%

\begin{Mfigure}{Cerenkov PID????}
  \centering
  \includegraphics[width=0.85\linewidth]{hms_cer_pid.pdf}
  \caption{Plot of HMS cerenkov histogram showing where an appropriate cut is applied.}
  \label{fig:3-3_hms_cer_pid}
\end{Mfigure}

\begin{Mfigure}{Calorimeter PID????}
  \centering
  \includegraphics[width=0.85\linewidth]{hms_cal_pid.pdf}
  \caption{Plot of HMS calorimeter histogram showing where an appropriate cut is applied.}
  \label{fig:3-3_hms_cal_pid}
\end{Mfigure}

% {"P_RF_Dist" : (P_RF_Dist > (0.75) ) & (P_RF_Dist < (1.75) )}, {"H_hod_goodstarttime" : (H_hod_goodstarttime == 1.0)}, {"P_hod_goodstarttime" : (P_hod_goodstarttime ==  1.0)}, {"P_hod_goodstarttime" : (P_hod_goodstarttime == 1.0)}, {"H_hod_goodstarttime" : (H_hod_goodstarttime == 1.0)}, {"H_gtr_dp" : ((H_gtr_dp > -8) & (H_gtr_dp < 8))},% {"P_gtr_dp" : ((P_gtr_dp > -10) & (P_gtr_dp < 20))}, {"P_gtr_beta" : ((abs(P_gtr_beta-1)) < 0.3)}, {"P_hgcer_npeSum" : (P_hgcer_npeSum < 1.5)}, {"P_aero_npeSum" : (P_aero_npeSum > 3)}, {"H_cal_etottracknorm" : (H_cal_etottracknorm > 0.7)}, {"P_cal_etottracknorm" : (P_cal_etottracknorm >= 0.0)}

\begin{table}[ht]
  \centering
  \begin{tabular}{cccc}
    \multicolumn{4}{c}{\large\textbf{Detector Cuts}} \\    
    \textbf{Detector/Particle} & \textbf{Gas Cerenkov [NPE]} & \textbf{Aerogel [NPE]} & \textbf{Calorimeter [$\mathbf{E_{cal}}$/E]} \\
    \hline
    HMS electron & >0.6??? & -    & >0.7     \\
    SHMS kaon    & <1.5    & >4.0 & $\ge$0.0 \\
    \multicolumn{4}{c}{\large\textbf{Acceptance Cuts}} \\
     & \textbf{$\mathbf{\delta}$ [\%]} & \textbf{$\mathbf{x'_{tar}}$} & \textbf{$\mathbf{y'_{tar}}$}  \\
    \hline
    HMS electron & - & ????  & ???? \\
    SHMS kaon    & - & ????  & ???? \\
    \multicolumn{4}{c}{\large\textbf{Timing Cuts}} \\
     & \textbf{$\mathbf{\left|\beta-1\right|}$}  & \textbf{RF} & \textbf{Cointime} \\
    \hline
    HMS electron & < 0.3 & ????          & ???? \\
    SHMS kaon    & < 0.3 &  $\pm$0.5???? & ???? \\
    \multicolumn{4}{c}{\large\textbf{MM Cuts}} \\
     & \textbf{$\mathbf{p(e, e'K^+)\Lambda}$}  &  & \\
    \hline
    HMS electron & ???? & & \\
    SHMS kaon    & ???? & & \\
  \end{tabular}
  \caption{}
  \label{tab:3-3_cuts}
\end{table}

\begin{Mfigure}{Reflector Rings}
  \centering
  \includegraphics[width=0.85\linewidth]{reflector_rings.pdf}
  \caption{Acrylic reflector rings applied to PMT 1 and 2 of the SHMS heavy gas cerenkov.}
  \label{fig:3-3_reflector_rings}
\end{Mfigure}

%\begin{Mfigure}{Reflector Cones}
%  \centecone
%  \includegraphics[width=0.85\linewidth]{reflector_cones.pdf}
%  \caption{AluMet reflector cones applied to all four PMTs of the SHMS heavy gas cerenkov.}
%  \label{fig:3-3_reflector_cones}
%\end{Mfigure}

\begin{figure}
  \centering
  \begin{minipage}[b]{0.48\linewidth}
    \includegraphics[width=\linewidth]{hgcer_hole_3233.pdf}
    \subcaption{Run 3233}
  \end{minipage}
  \hfill
  \begin{minipage}[b]{0.48\linewidth}
    \includegraphics[width=\linewidth]{hgcer_hole_4787.pdf}
    \subcaption{Run 4787}
  \end{minipage}
  
  \vspace{0.5cm}
  
  \begin{minipage}[b]{0.48\linewidth}
    \includegraphics[width=\linewidth]{hgcer_hole_6619.pdf}
    \subcaption{Run 6619}
  \end{minipage}
  \hfill
  \begin{minipage}[b]{0.48\linewidth}
    \includegraphics[width=\linewidth]{hgcer_hole_7095.pdf}
    \subcaption{Run 7095}
  \end{minipage}
  
  \caption{Comparison of the SHMS heavy gas cerenkov hole for runs 3233 (a), 4783 (b), 6619 (c), and 7095 (d). Run 3233 is the original HGCer configuration, run 4783 is after the first mirror adjustments and the optics replaced with an Acrylic ring, run 6619 is with the cones installed, and run 7095 is with the second alignment and with the cones installed. All runs are for electron PID (Calorimeter cut > 0.9) and with consistent calibrations. The NPE increased with each iteration, but neither alignment decreased the hole back to its original size.}
  \label{fig:3-3_hgcer_hole}
\end{figure}

\begin{Mfigure}{EDTM?????}
  \centering
  \includegraphics[width=0.85\linewidth]{edtm_runnum.pdf}
  \caption{Total Live Times of all production data.}
  \label{fig:3-4_edtm_runnum}
\end{Mfigure}

\begin{Mfigure}{Scaler, no track, and track luminosity yield?????}
  \centering
  \includegraphics[width=0.85\linewidth]{carbon_lumi_yield.pdf}
  \caption{The HMS Carbon-12 scaler (left), no track (middle) and track (right) yields for runs taken at $P_{HMS}$=-3.09, $\theta_{HMS}$=35.010.}
  \label{fig:3-4_carbon_lumi_yield}
\end{Mfigure}

\begin{Mfigure}{HMS Linear Regression?????}
  \centering
  \includegraphics[width=0.85\linewidth]{hms_linear_regress.pdf}
  \caption{All HMS Carbon-12 fit with a weighted linear regression using least squares.}
  \label{fig:3-4_hms_linear_regress}
\end{Mfigure}

\begin{Mfigure}{SHMS Linear Regression?????}
  \centering
  \includegraphics[width=0.85\linewidth]{shms_linear_regress.pdf}
  \caption{All SHMS Carbon-12 fit with a weighted linear regression using least squares.}
  \label{fig:3-4_shms_linear_regress}
\end{Mfigure}

%%%%%%%%%%%%%%%%%%%%%%%%%%%%%%%%%%%%%%%%%%%%%%%%%%%%%%%%%%%%%%%%%%%%%%%%%%%%%%%%%%%%%%%%%%%%%%%%%%%%%%%%%%%%%%%%%%%%%%%%
%
% Chapter 4, section 5
%
%%%%%%%%%%%%%%%%%%%%%%%%%%%%%%%%%%%%%%%%%%%%%%%%%%%%%%%%%%%%%%%%%%%%%%%%%%%%%%%%%%%%%%%%%%%%%%%%%%%%%%%%%%%%%%%%%%%%%%%%

\begin{Mtable}{Target Loop}
  \centering
  \begin{tabular}{|c|c|}
    \hline
    \textbf{Decay Channel} & \textbf{Branching Fracion} \\
    \hline
    $K^+\rightarrow \mu^++\nu_{\mu}$ & 63.55\% \\
    $K^+\rightarrow \pi^++\pi^0$ & 20.66\% \\
    $K^+\rightarrow \pi^++\pi^++\pi^0$ & 5.59\% \\
    $K^+\rightarrow e+\pi^0+\nu_e$ & 5.07\% \\
    $K^+\rightarrow \mu^++\pi^0+\nu_{\mu}$ & 3.35\% \\
    $K^+\rightarrow \pi^++\pi^0+\pi^0$ & 1.76\% \\
    \hline
    \end{tabular}
  \caption{Most probable decay channels of $K^+$ and their respective branching fractions. There are many decay channels for the kaon but these make up >99.98\% of the kaon decay's phase space.}
  \label{tab:4-5_kaon_decay}
\end{Mtable}
