%%%%%%%%%% Example %%%%%%%%%%

%\begin{Mfigure}{Lorem Ipsum}
% \centering
% \includegraphics[width=\linewidth]{Figure-C1.pdf}
% \caption[\figuretitle]
{\textbf{(A) Lorem Ipsum.} Add as much text here as you want.}
%\label{fig:c1}
%\end{Mfigure}

\begin{Mfigure}{CUA Logo}
 \centering
 \includegraphics[width=0.5\linewidth]{CUA_logo.pdf}
\label{fig:titlepagelogo}
\end{Mfigure}

%%%%%%%%%%%%%%%%%%%%%%%%%%%%%%%%%%%%%%%%%%%%%%%%%%%%%%%%%%%%%%%%%%%%%%%%%%%%%%%%%%%%%%%%%%%%%%%%%%%%%%%%%%%%%%%%%%%%%%%%
%
% Chapter 1, section 6
%
%%%%%%%%%%%%%%%%%%%%%%%%%%%%%%%%%%%%%%%%%%%%%%%%%%%%%%%%%%%%%%%%%%%%%%%%%%%%%%%%%%%%%%%%%%%%%%%%%%%%%%%%%%%%%%%%%%%%%%%%

\begin{Mfigure}{Mass Aquisition of Quarks}
  \centering
  \includegraphics[width=0.75\linewidth]{dressquarkmass.pdf}
  \caption{The dressed-quark mass function, M(p), versus momentum which shows the rapid aquisition of mass via the gluon cloud as a result of DCSB. The DSE results are the solid curves and the pseudodata points are produced from numerical simulations of lattice-regularized QCD \cite{roberts_hadron_2015}.}
  \label{fig:1-1_dressquarkmass}
\end{Mfigure}

\begin{Mfigure}{Sullivan Process}
  \centering
  \includegraphics[width=0.75\linewidth]{sullivanprocess.pdf}
  \caption{Diagram showing the Sullivan Process for acessing the pion's (a) elastic form factor and (b) parton distribution functions (PDFs) \cite{qin_off-shell_2018}.}
  \label{fig:1-1_sullivanprocess}
\end{Mfigure}

\begin{Mfigure}{p(e,e'K+)lambda}
  \centering
  \includegraphics[width=0.75\linewidth]{reactionplane.pdf}
  \caption{Diagram of reaction and scattering planes for $p(e,e'K^2)\Lambda$ reaction.}
  \label{fig:1-1_reactionplane}
\end{Mfigure}

\begin{Mfigure}{mandelstam}
  \centering
  \includegraphics[width=0.75\linewidth]{mandelstam.pdf}
  \caption{Diagram comparing the different Mandelstam channels.}
  \label{fig:1-1_mandelstam}
\end{Mfigure}

\begin{Mtable}{Kinematics}
  \centering
  \begin{tabular}{|c|c|c|c|c|c|c|c|c|}
    \hline
    \textbf{E (GeV)} & \textbf{$Q^2$ ($GeV^2$)} & \textbf{W (GeV)} & \textbf{x} & \textbf{$\epsilon_{high}$/$\epsilon_{low}$} & \textbf{$\Delta\epsilon$} & \textbf{Study Type}\\
    \hline
    10.6/8.2 & 5.5 & 3.02 & 0.40 & 0.53/0.18 & 0.35 & Scaling \\
    10.6/8.2 & 4.4 & 2.74 & 0.40 & 0.72/0.48 & 0.24 & Scaling \\
    10.6/8.2 & 3.0 & 3.14 & 0.25 & 0.67/0.39 & 0.28 & Scaling \\
    10.6/6.2 & 3.0 & 2.32 & 0.40 & 0.88/0.57 & 0.31 & Scaling+FF \\
    10.6/6.2 & 2.115 & 2.95 & 0.21 & 0.79/0.25 & 0.54 & Scaling+FF \\
    \hline
    \end{tabular}
  \caption{Summary of all kinematics for the KaonLT 2018-19 experiment.}
  \label{tab:1-1_kinematics}
\end{Mtable}

%%%%%%%%%%%%%%%%%%%%%%%%%%%%%%%%%%%%%%%%%%%%%%%%%%%%%%%%%%%%%%%%%%%%%%%%%%%%%%%%%%%%%%%%%%%%%%%%%%%%%%%%%%%%%%%%%%%%%%%%
%
% Chapter 2, section 1
%
%%%%%%%%%%%%%%%%%%%%%%%%%%%%%%%%%%%%%%%%%%%%%%%%%%%%%%%%%%%%%%%%%%%%%%%%%%%%%%%%%%%%%%%%%%%%%%%%%%%%%%%%%%%%%%%%%%%%%%%%

\begin{Mfigure}{Schematic of CEBAF 12 GeV Upgrade}
  \centering
  \includegraphics[width=0.75\linewidth]{cebaf_12gev.pdf}
  \caption{Schematic of CEBAF 12 GeV Upgrade.}
  \label{fig:2-1_cebaf12gev}
\end{Mfigure}

%%%%%%%%%%%%%%%%%%%%%%%%%%%%%%%%%%%%%%%%%%%%%%%%%%%%%%%%%%%%%%%%%%%%%%%%%%%%%%%%%%%%%%%%%%%%%%%%%%%%%%%%%%%%%%%%%%%%%%%%
%
% Chapter 2, section 2
%
%%%%%%%%%%%%%%%%%%%%%%%%%%%%%%%%%%%%%%%%%%%%%%%%%%%%%%%%%%%%%%%%%%%%%%%%%%%%%%%%%%%%%%%%%%%%%%%%%%%%%%%%%%%%%%%%%%%%%%%%

\begin{Mfigure}{Hall C Arc to Beamline}
  \centering
  \includegraphics[width=0.75\linewidth]{hallc_arc.pdf}
  \caption{The Hall C arc which steers the beam to the beamline. Electron synchrotron radiation loss is shown with the yellow arrows.}
  \label{fig:2-2_hallc_arc}
\end{Mfigure}

\begin{Mfigure}{Hall C Beamline}
  \centering
  \includegraphics[width=0.75\linewidth]{beamline.pdf}
  \caption{Hall C beamline from entrance of hall to the target scattering chamber.}
  \label{fig:2-2_beamline}
\end{Mfigure}

\begin{Mfigure}{Hall C Beamline Components}
  \centering
  \includegraphics[width=0.75\linewidth]{beamline_components.pdf}
  \caption{The Hall C beamline components from entrance of hall to the target scattering chamber. The relevant distances from the scattering chamber are labeled.}
  \label{fig:2-2_beamline_components}
\end{Mfigure}

\begin{Mfigure}{Hall C Beamline Big BPMS}
  \centering
  \includegraphics[width=0.75\linewidth]{beamline_bigbpms.pdf}
  \caption{Schematic of the Hall C beamline downstream, showing the Big BPMs (IPM3H08, IPM3H09) near the beam dump..}
  \label{fig:2-2_beamline_bigbpms}
\end{Mfigure}

\begin{Mfigure}{XY Raster Plot}
  \centering
  \includegraphics[width=0.75\linewidth]{raster.pdf}
  \caption{Fast raster XY plot of run 6881 during 4.9 GeV. The uniform beam dispersion is evident by the teal homogeneity which corresponds to ~15 events per bin.}
  \label{fig:2-2_raster}
\end{Mfigure}

\begin{Mfigure}{BCM1/2 and Unser}
  \centering
  \includegraphics[width=0.75\linewidth]{bcm_unser.pdf}
  \captionsetup{width=0.95\textwidth} % Adjust the caption width here
  \caption{Drawing showing BCM1 and BCM2 on either side of the Unser monitor. A Unser wire calibration can be performed by passing a wire (labeled calibration wire), with an accurately known current, through the Unser monitor to measure the gain of the electronic chain.}
  \label{fig:2-2_bcm_unser}
\end{Mfigure}

%%%%%%%%%%%%%%%%%%%%%%%%%%%%%%%%%%%%%%%%%%%%%%%%%%%%%%%%%%%%%%%%%%%%%%%%%%%%%%%%%%%%%%%%%%%%%%%%%%%%%%%%%%%%%%%%%%%%%%%%
%
% Chapter 2, section 3
%
%%%%%%%%%%%%%%%%%%%%%%%%%%%%%%%%%%%%%%%%%%%%%%%%%%%%%%%%%%%%%%%%%%%%%%%%%%%%%%%%%%%%%%%%%%%%%%%%%%%%%%%%%%%%%%%%%%%%%%%%

\begin{Mfigure}{Target Loop}
  \centering
  \includegraphics[width=0.75\linewidth]{target_loop.pdf}
  \caption{ CAD view of the cryogenic and solid target ladders. There are three cyrogenic loops above the solid target ladder.}
  \label{fig:2-3_target_loop}
\end{Mfigure}

\begin{Mtable}{Target Loop}
  \centering
  \begin{tabular}{|c|c|c|c|}
    \hline
    \textbf{Loop Number} & \textbf{Target} & \textbf{Length} & \textbf{Current Limits} \\
    \hline
    2 & L$H_2$ & 10 cm & ???? \\
    - & Dummy & 10 cm  & ???? \\
    - & Optics 1 & 20 cm  & ???? \\
    - & Carbon Hole & ???  & ???? \\
    - & Carbon 1.5\% & ???  & ???? \\
    \hline
    \end{tabular}
  \caption{Break down of all targets available during the running period. The three cryogenic targets are labeled alongside their corresponding loops. The target length is also provided for each type.}
  \label{tab:2-3_target_loop}
\end{Mtable}

\begin{Mtable}{LH2 Information}
  \centering
  \begin{tabular}{|c|c|c|c|}
    \hline
    \textbf{Target} & \textbf{Target Density} & \textbf{Target Length} & \textbf{Target Thickness} \\
    \hline
    $LH_2$ & 0.07231 g/$cm^3$ & 10 cm & mm \\
    \hline
    \end{tabular}
  \label{tab:2-3_lh2_properties}
\end{Mtable}

%%%%%%%%%%%%%%%%%%%%%%%%%%%%%%%%%%%%%%%%%%%%%%%%%%%%%%%%%%%%%%%%%%%%%%%%%%%%%%%%%%%%%%%%%%%%%%%%%%%%%%%%%%%%%%%%%%%%%%%%
%
% Chapter 2, section 4
%
%%%%%%%%%%%%%%%%%%%%%%%%%%%%%%%%%%%%%%%%%%%%%%%%%%%%%%%%%%%%%%%%%%%%%%%%%%%%%%%%%%%%%%%%%%%%%%%%%%%%%%%%%%%%%%%%%%%%%%%%

\begin{Mtable}{Spectrometer Specifications}
  \centering
  \begin{tabular}{|c|c|c|c|}
    \hline
    \textbf{Parameter} & \textbf{HMS} & \textbf{SHMS}\\
    \hline
    Central Momentum & 0.4 - 7.4 GeV/c & 2 - 11 GeV/c \\
    Momentum Acceptance & $\pm$10\% & -10\% - +22\% \\
    Momentum Resolution & 0.1\% - 0.15\% & 0.03\% - 0.08\% \\
    Scattering Angle Range & 10.5$\degree$ - 90$\degree$ & 5.5$\degree$ - 40$\degree$ \\
    \hline
    Horizontal Angle Acceptance & $\pm$32 mrad & $\pm$18 mrad \\
    Horizontal Angle Resolution & 0.8 mrad & 0.5 - 1.2 mrad \\
    Vertical Angle Acceptance & $\pm$85 mrad & $\pm$50 mrad \\
    Vertical Angle Resolution & 1.0 mrad & 0.3 - 1.1 mrad \\
    Solid Angle Acceptance & 8.1 msr & >4 msr \\
    \hline
    Maximum Event Rate & 2000 Hz & 10,000 Hz \\
    e/h Discrimination & >1000:1 at 98\% efficiency & >1000:1 at 98\% efficiency \\
    $\pi$/K Discrimination & 100:1 at 95\% efficiency & 100:1 at 95\% efficiency \\
    \hline
  \end{tabular}
  \caption{Break down of the HMS and SHMS specifications and capablilities..}  
  \label{tab:2-4_spectrometer}
\end{Mtable}

\begin{Mfigure}{Spectrometer Setup}
  \centering
  \includegraphics[width=0.75\linewidth]{spec_setup.pdf}
  \caption{Overview of Hall C Spectrometer setup.}
  \label{fig:2-4_spec_setup}
\end{Mfigure}

\begin{Mfigure}{HMS Magnets}
  \centering
  \includegraphics[width=0.75\linewidth]{hms_magnets.pdf}
  \caption{Overview of HMS optical setup.}
  \label{fig:2-4_hms_magnets}
\end{Mfigure}

\begin{Mtable}{HMS Slit System}
  \centering
  \begin{tabular}{|c|c|c|c|}
    \hline
    \textbf{Collimator} & \textbf{d$\Omega$ (msr)} & \textbf{Horizontal (mr)} & \textbf{Vertical (mr)} \\
    \hline
    Large Collimator & 6.74 & $\pm$27.5 & $\pm$70.0 \\
    Pion Collimator & 4.03 & $\pm$21.3 & $\pm$54.0 \\
    \hline
  \end{tabular}
  \caption{Breakdown of HMS slit system's apertures.}
  \label{tab:2-4_hms_slit}
\end{Mtable}

\begin{Mfigure}{SHMS Magnets}
  \centering
  \includegraphics[width=0.75\linewidth]{shms_magnets.pdf}
  \caption{Overview of SHMS optical setup.}
  \label{fig:2-4_shms_magnets}
\end{Mfigure}

\begin{Mtable}{SHMS Slit System}
  \centering
  \begin{tabular}{|c|c|c|c|}
    \hline
    \textbf{Collimator} & \textbf{d$\Omega$ (msr)} & \textbf{Horizontal (mr)} & \textbf{Vertical (mr)} \\
    \hline
     Collimator & 4.00 & $\pm$24.0 & $\pm$40.0 \\
    \hline
  \end{tabular}
  \caption{Breakdown of SHMS slit system's apertures.}
  \label{tab:2-4_shms_slit}
\end{Mtable}

\begin{Mfigure}{HMS Detector Stack}
  \centering
  \includegraphics[width=0.75\linewidth]{hms_detectors.pdf}
  \caption{Overview of HMS Detector Stack.}
  \label{fig:2-4_hms_detectors}
\end{Mfigure}

\begin{Mfigure}{SHMS Detector Stack}
  \centering
  \includegraphics[width=0.75\linewidth]{shms_detectors.pdf}
  \caption{Overview of SHMS Detector Stack.}
  \label{fig:2-4_shms_detectors}
\end{Mfigure}

\begin{Mfigure}{Drift Chamber View}
  \centering
  \includegraphics[width=0.75\linewidth]{dc_view.pdf}
  \caption{Basic design and components of DC1 (left) and DC2 (right).}
  \label{fig:2-4_dc_view}
\end{Mfigure}

\begin{Mfigure}{Hodoscope View}
  \centering
  \includegraphics[width=0.75\linewidth]{hodo_view.pdf}
  \caption{Basic design of a hodoscope pair.}
  \label{fig:2-4_hodo_view}
\end{Mfigure}

\begin{Mfigure}{SHMS HGC View}
  \centering
  \includegraphics[width=0.75\linewidth]{shms_hgc.pdf}
  \caption{View of the SHMS Heavy Gas Cerenkov's four mirrors and PMTs.}
  \label{fig:2-4_shms_hgc}
\end{Mfigure}

\begin{Mfigure}{SHMS Aerogel View}
  \centering
  \includegraphics[width=0.75\linewidth]{shms_aero.pdf}
  \caption{View of the SHMS aerogel Cerenkov.}
  \label{fig:2-4_shms_aero}
\end{Mfigure}

\begin{Mtable}{SHMS Aerogel Threshold Momenta}
  \centering
  \begin{tabular}{|c|c|c|c|c|}
    \hline
    \textbf{Particle} & \textbf{$P_{Th}$} & \textbf{$P_{Th}$} & \textbf{$P_{Th}$} & \textbf{$P_{Th}$} \\
    \hline
    $\mu$ & 0.428 & 0.526 & 0.608 & 0.711 \\
    $\pi$ & 0.565 & 0.692 & 0.803 & 0.935 \\
    $K$ & 2.000 & 2.453 & 2.840 & 3.315 \\
    $p$ & 3.802 & 4.667 & 5.379 & 6.307 \\
    \hline
  \end{tabular}
  \caption{Threshold momenta ($P_{Th}$ in GeV/c) for a variety of charged particles and the corresponding refractive indices.}
  \label{tab:2-4_aero_threshold}
\end{Mtable}

\begin{Mfigure}{SHMS Aerogel Tray Exchange}
  \centering
  \includegraphics[width=0.5\linewidth]{aero_tray.pdf}
  \caption{View of an aerogel tray being lifted out of the SHMS hut. The tray maintains its 18$\degree$ angle until it is slowly lowerd on the pallet.}
  \label{fig:2-4_aero_tray}
\end{Mfigure}

\begin{Mfigure}{SHMS Calorimeter}
  \centering
  \includegraphics[width=0.75\linewidth]{shms_cal.pdf}
  \caption{View of the SHMS calorimeter which shows the preshow and shower, along with their corresponding lead glass blocks.}
  \label{fig:2-4_shms_cal}
\end{Mfigure}

%%%%%%%%%%%%%%%%%%%%%%%%%%%%%%%%%%%%%%%%%%%%%%%%%%%%%%%%%%%%%%%%%%%%%%%%%%%%%%%%%%%%%%%%%%%%%%%%%%%%%%%%%%%%%%%%%%%%%%%%
%
% Chapter 2, section 5
%
%%%%%%%%%%%%%%%%%%%%%%%%%%%%%%%%%%%%%%%%%%%%%%%%%%%%%%%%%%%%%%%%%%%%%%%%%%%%%%%%%%%%%%%%%%%%%%%%%%%%%%%%%%%%%%%%%%%%%%%%

\begin{Mfigure}{SHMS Trigger and DAQ}
  \centering
  \includegraphics[width=0.75\linewidth]{shms_trigger.pdf}
  \caption{Overview of the SHMS trigger and DAQ system. The HMS shares a similar trigger and DAQ system.}
  \label{fig:2-5_shms_trigger}
\end{Mfigure}

\begin{Mtable}{Hall C Pre-triggers}
  \centering
  \begin{tabular}{|c|c|}
    \hline
    \textbf{Input Triggers} & \textbf{Pre-trigger} \\
    \hline    
    \text{pTRIG1} & \text{pHODO 3/4} \\
    \text{pTRIG2} & \text{pEL REAL} \\
    \text{pTRIG3} & \text{hEL REAL} \\
    \text{pTRIG4} & \text{hHODO 3/4} \\
    \text{pTRIG5} & \text{hEL REAL + pHODO 3/4} \\
    \text{pTRIG6} & \text{hEL REAL + pEL REAL} \\
    \hline
  \end{tabular}
  \caption{The input triggers of the TM (i.e. ROC 02) compared to their corresponding single-arm ptr-trigger combinations.}
  \label{tab:2-5_pretriggers}
\end{Mtable}

\begin{Mtable}{Hall C Pre-triggers}
  \centering
  \begin{tabular}{|c|c|}
    \hline    
    \text{reference time}  = \text{TDC Start, single-arm pre-trigger} - \text{TDC Stop, L1ACCP} \\
    \text{detector raw time}  = \text{TDC Start, detector signal} - \text{TDC Stop, L1ACCP} \\
    \text{detector time}  = \text{detector raw time} - \text{reference time} \\
    \hline    
  \end{tabular}
  \label{tab:2-5_ref_sub}
\end{Mtable}

\begin{Mtable}{Hall C Pre-triggers}
  \centering
  \begin{tabular}{|c|c|c|}
    \hline
    \textbf{Value} & \textbf{Prescale Factor} & \textbf{Rate Prescaled from 100 kHz} \\
    \hline    
    0   &        1 & 100000 \\
    1   &        2 & 50000 \\
    2   &        3 & 33333 \\
    3   &        5 & 20000 \\
    4   &        9 & 11111 \\
    5   &       17 & 5882 \\
    6   &       33 & 3030 \\
    7   &       65 & 1538 \\
    8   &      129 & 775 \\
    9   &      257 & 389 \\
    0   &      513 & 194 \\
    1   &     1025 & 97 \\
    2   &     2049 & 48 \\
    3   &     4097 & 24 \\
    4   &     8193 & 12 \\
    5   &    16385 & 6 \\
    6   &    32769 & 3 \\
    \hline
  \end{tabular}
  \caption{List of prescale values to their corresponding factor. The third column shows an example of a prescaled rate from 100 kHz.}
  \label{tab:2-5_prescale}
\end{Mtable}


%%%%%%%%%%%%%%%%%%%%%%%%%%%%%%%%%%%%%%%%%%%%%%%%%%%%%%%%%%%%%%%%%%%%%%%%%%%%%%%%%%%%%%%%%%%%%%%%%%%%%%%%%%%%%%%%%%%%%%%%
%
% Chapter 3, section 1
%
%%%%%%%%%%%%%%%%%%%%%%%%%%%%%%%%%%%%%%%%%%%%%%%%%%%%%%%%%%%%%%%%%%%%%%%%%%%%%%%%%%%%%%%%%%%%%%%%%%%%%%%%%%%%%%%%%%%%%%%%

\begin{Mfigure}{LT Analysis Procedure}
  \centering
  \includegraphics[width=0.85\linewidth]{lt_analysis_procedure.pdf}
  \caption{LT analysis procedure.}
  \label{fig:3-1_lt_analysis_procedure}
\end{Mfigure}

\begin{landscape}
\begin{Mfigure}{Stage 1 of Analysis Procedure}
  \centering
  \includegraphics[width=1.1\linewidth]{LT_Analysis_Workflow_1.pdf}
  \caption{Preprocess the data by applying offsets and defining PID and CT cuts.}
  \label{fig:3-1_LT_Analysis_Workflow_1}
\end{Mfigure}
\end{landscape}

\begin{landscape}
\begin{Mfigure}{Stage 2 of Analysis Procedure}
  \centering
  \includegraphics[width=1.1\linewidth]{LT_Analysis_Workflow_2.pdf}
  \caption{Setup for analysis procedure by applying cuts, running SIMC, combining data per setting, and calculating the effective charge.} 
  \label{fig:3-1_LT_Analysis_Workflow_2}
\end{Mfigure}
\end{landscape}

\begin{landscape}
\begin{Mfigure}{Stage 3 of Analysis Procedure}
  \centering
  \includegraphics[width=1.1\linewidth]{LT_Analysis_Workflow_3.pdf}
  \caption{Analyze data by limiting to kinematically accessible region and within proper acceptances, subtracting randoms and leak-though, and, finally, finding the yield and average kinematic values per bin..}
  \label{fig:3-1_LT_Analysis_Workflow_3}
\end{Mfigure}
\end{landscape}

\begin{landscape}
\begin{Mfigure}{Stage 4 of Analysis Procedure}
  \centering
  \includegraphics[width=1.1\linewidth]{LT_Analysis_Workflow_4.pdf}
  \caption{Calculate the unseparated cross section and then perform the LT-Separation to obtain separated cross sections..}
  \label{fig:3-1_LT_Analysis_Workflow_4}
\end{Mfigure}
\end{landscape}

\begin{landscape}
\begin{Mfigure}{Stage 5 of Analysis Procedure}
  \centering
  \includegraphics[width=1.1\linewidth]{LT_Analysis_Workflow_5.pdf}
  \caption{Iterate on the model to optimize the data to SIMC comparison.}
  \label{fig:3-1_LT_Analysis_Workflow_5}
\end{Mfigure}
\end{landscape}


%%%%%%%%%%%%%%%%%%%%%%%%%%%%%%%%%%%%%%%%%%%%%%%%%%%%%%%%%%%%%%%%%%%%%%%%%%%%%%%%%%%%%%%%%%%%%%%%%%%%%%%%%%%%%%%%%%%%%%%%
%
% Chapter 3, section 3
%
%%%%%%%%%%%%%%%%%%%%%%%%%%%%%%%%%%%%%%%%%%%%%%%%%%%%%%%%%%%%%%%%%%%%%%%%%%%%%%%%%%%%%%%%%%%%%%%%%%%%%%%%%%%%%%%%%%%%%%%%

\begin{Mfigure}{HMS Cerenkov PID}
  \centering
  \includegraphics[width=0.85\linewidth]{hms_cer_pid.pdf}
  \caption{HMS cerenkov vs Missing Mass showing the $\Lambda$ peak at ~1.115 $\mathrm{GeV}$. There is a HMS cerenkov cut of >1.5 applied. This is an example from $Q^2=2.1$, $W=2.95$ low epsilon, center setting.}
  \label{fig:3-3_hms_cer_pid}
\end{Mfigure}

\begin{Mfigure}{HMS Calorimeter PID}
  \centering
  \includegraphics[width=0.85\linewidth]{hms_cal_pid.pdf}
  \caption{HMS calorimeter vs Missing Mass showing the $\Lambda$ peak at ~1.115 $\mathrm{GeV}$. There is a HMS calorimeter cut of >0.7 applied. This is an example from $Q^2=2.1$, $W=2.95$ low epsilon, center setting.}
  \label{fig:3-3_hms_cal_pid}
\end{Mfigure}

\begin{Mfigure}{SHMS Aerogel PID}
  \centering
  \includegraphics[width=0.85\linewidth]{shms_aero_pid.pdf}
  \caption{SHMS aerogel vs Missing Mass showing the $\Lambda$ peak at ~1.115 $\mathrm{GeV}$. There is a SHMS aerogel cut of >3 applied for clean $K^+$-$p$ separation. This is an example from $Q^2=2.1$, $W=2.95$ low epsilon, center setting.}
  \label{fig:3-3_shms_aero_pid}
\end{Mfigure}

\begin{Mfigure}{SHMS Calorimeter PID}
  \centering
  \includegraphics[width=0.85\linewidth]{shms_cal_pid.pdf}
  \caption{SHMS calorimeter vs Missing Mass showing the $\Lambda$ peak at ~1.115 $\mathrm{GeV}$. There is a SHMS calorimeter cut of $\geq0.0$ applied. This is an example from $Q^2=2.1$, $W=2.95$ low epsilon, center setting.}
  \label{fig:3-3_shms_cal_pid}
\end{Mfigure}

\begin{table}[ht]
  \centering
  \begin{tabular}{cccc}
    \multicolumn{4}{c}{\large\textbf{Detector Cuts}} \\    
    \textbf{Detector/Particle} & \textbf{Gas Cerenkov [NPE]} & \textbf{Aerogel [NPE]} & \textbf{Calorimeter [$\mathbf{E_{cal}}$/E]} \\
    \hline
    HMS electron & >1.5 & -    & >0.7     \\
    SHMS kaon    & <1.5    & >3.0 & $\ge$0.0 \\
    \multicolumn{4}{c}{\large\textbf{Acceptance Cuts}} \\
     & \textbf{$\mathbf{\delta}$ [\%]} & \textbf{$\mathbf{x'_{tar}}$} & \textbf{$\mathbf{y'_{tar}}$}  \\
    \hline
    HMS electron & $-8.0\geq\delta\geq8.0$ & $|x'_{tar}|\geq0.08$  & $|y'_{tar}|\geq0.045$ \\
    SHMS kaon    & $-10.0\geq\delta\geq20.0$ & $|x'_{tar}|\geq0.06$  & $|y'_{tar}|\geq0.04$ \\
    \multicolumn{4}{c}{\large\textbf{Timing Cuts}} \\
     & \textbf{$\mathbf{\left|\beta-1\right|}$}  & \textbf{RF} & \textbf{Cointime} \\
    \hline
    $e-K^+$    & < 0.3 &  $0.75<RF<1.75$ & $|CT|<1.754$ \\
    \multicolumn{4}{c}{\large\textbf{MM Cuts}} \\
     & \textbf{$\mathbf{p(e, e'K^+)\Lambda}$}  &  & \\
    \hline
    $p(e,e'K^+)\Lambda$ & $1.10<MM<1.24$ & & \\
  \end{tabular}
  \caption{}
  \label{tab:3-3_cuts}
\end{table}

\begin{Mfigure}{Reflector Rings}
  \centering
  \includegraphics[width=0.85\linewidth]{reflector_rings.pdf}
  \caption{Acrylic reflector rings applied to PMT 1 and 2 of the SHMS heavy gas cerenkov.}
  \label{fig:3-3_reflector_rings}
\end{Mfigure}

\begin{figure}
  \centering
  \begin{minipage}[b]{0.48\linewidth}
    \includegraphics[width=\linewidth]{hgcer_hole_3233.pdf}
    \subcaption{Run 3233}
  \end{minipage}
  \hfill
  \begin{minipage}[b]{0.48\linewidth}
    \includegraphics[width=\linewidth]{hgcer_hole_4787.pdf}
    \subcaption{Run 4787}
  \end{minipage}
  
  \vspace{0.5cm}
  
  \begin{minipage}[b]{0.48\linewidth}
    \includegraphics[width=\linewidth]{hgcer_hole_6619.pdf}
    \subcaption{Run 6619}
  \end{minipage}
  \hfill
  \begin{minipage}[b]{0.48\linewidth}
    \includegraphics[width=\linewidth]{hgcer_hole_7095.pdf}
    \subcaption{Run 7095}
  \end{minipage}
  
  \caption{Comparison of the SHMS heavy gas cerenkov hole for runs 3233 (a), 4783 (b), 6619 (c), and 7095 (d). Run 3233 is the original HGCer configuration, run 4783 is after the first mirror adjustments and the optics replaced with an Acrylic ring, run 6619 is with the cones installed, and run 7095 is with the second alignment and with the cones installed. All runs are for electron PID (Calorimeter cut > 0.9) and with consistent calibrations. The NPE increased with each iteration, but neither alignment decreased the hole back to its original size.}
  \label{fig:3-3_hgcer_hole}
\end{figure}

\begin{figure}
  \centering
  \begin{minipage}[b]{0.48\linewidth}
    \includegraphics[width=\linewidth]{data_hgcer_hole_nocut.pdf}
    \subcaption{Data without hole cut}
  \end{minipage}
  \hfill
  \begin{minipage}[b]{0.48\linewidth}
    \includegraphics[width=\linewidth]{data_hgcer_hole_cut.pdf}
    \subcaption{Data with hole cut}
  \end{minipage}
  
  \vspace{0.5cm}
  
  \begin{minipage}[b]{0.48\linewidth}
    \includegraphics[width=\linewidth]{simc_hgcer_hole_nocut.pdf}
    \subcaption{SIMC without hole cut}
  \end{minipage}
  \hfill
  \begin{minipage}[b]{0.48\linewidth}
    \includegraphics[width=\linewidth]{simc_hgcer_hole_cut.pdf}
    \subcaption{SIMC with hole cut}
  \end{minipage}
  
  \caption{Comparison of the SHMS heavy gas cerenkov hole cut for data and SIMC. This is an example from $Q^2=3.0$, $W=3.14$ low epsilon, center setting.}
  \label{fig:3-3_hgcer_hole_cut}
\end{figure}

\begin{Mfigure}{Coin time vs Beta}
  \centering
  \includegraphics[width=0.85\linewidth]{cointime_beta.pdf}
  \caption{$e-K^+$ Coin time vs beta. There is a coin time cut of $|CT|<1.754$ and beta cut of $|\beta-1|<0.3$ applied. This is an example from $Q^2=4.4$, $W=2.74$ high epsilon, center setting.}
  \label{fig:3-3_cointime_beta}
\end{Mfigure}

\begin{Mfigure}{Coin time vs MM}
  \centering
  \includegraphics[width=0.85\linewidth]{cointime_MM.pdf}
  \caption{$e-K^+$ Coin time vs Missing Mass. There is a coin time cut of $|CT|<1.754$. This is an example from $Q^2=4.4$, $W=2.74$ high epsilon, center setting.}
  \label{fig:3-3_cointime_MM}
\end{Mfigure}

\begin{Mfigure}{MM spectrum}
  \centering
  \includegraphics[width=0.85\linewidth]{mm_spectrum.pdf}
  \caption{Missing mass spectrum without subtraction for $Q^2=3.0$, $W=3.14$ low epsilon. This spectrum shows the pion leak-through in the lower MM range as well as the higher $K^+$ channels and SIDIS background.}
  \label{fig:3-3_mm_spectrum}
\end{Mfigure}

\begin{Mfigure}{MM spectrum cut}
  \centering
  \includegraphics[width=0.85\linewidth]{mm_spectrum_cut.pdf}
  \caption{Missing mass spectrum with $1.10<MM<1.24$ cut applied for $Q^2=3.0$, $W=3.14$ low epsilon.}
  \label{fig:3-3_mm_spectrum_cut}
\end{Mfigure}

\begin{Mfigure}{Aerogel Efficiency}
  \centering
  \includegraphics[width=0.85\linewidth]{paero_eff.pdf}
  \caption{Aerogel efficiency across the highest beam energy of $10.6\mathrm{GeV}$.}
  \label{fig:3-3_paero_eff}
\end{Mfigure}

\begin{Mfigure}{Electron Tracking Efficiency}
  \centering
  \includegraphics[width=0.85\linewidth]{e_track.pdf}
  \caption{Electron tracking efficiency across the highest beam energy of $10.6\mathrm{GeV}$.}
  \label{fig:3-3_e_track}
\end{Mfigure}

\begin{Mfigure}{Pion Tracking Efficiency}
  \centering
  \includegraphics[width=0.85\linewidth]{k_track.pdf}
  \caption{Pion tracking efficiency across the highest beam energy of $10.6\mathrm{GeV}$. The pion tracking was used because of the low statistics for kaon. They have similar ToF so there is little difference in the final tracking efficiencies between the particles.}
  \label{fig:3-3_k_track}
\end{Mfigure}

\begin{Mfigure}{EDTM}
  \centering
  \includegraphics[width=0.85\linewidth]{edtm_runnum.pdf}
  \caption{Total Live Times of all production data.}
  \label{fig:3-4_edtm_runnum}
\end{Mfigure}

\begin{Mfigure}{Scaler, no track, and track luminosity yield}
  \centering
  \includegraphics[width=0.85\linewidth]{carbon_lumi_yield.pdf}
  \caption{The HMS Carbon-12 scaler (left), no track (middle) and track (right) yields for runs taken at $P_{HMS}$=-3.09, $\theta_{HMS}$=35.010.}
  \label{fig:3-4_carbon_lumi_yield}
\end{Mfigure}

\begin{Mfigure}{HMS Linear Regression}
  \centering
  \includegraphics[width=0.85\linewidth]{hms_linear_regress.pdf}
  \caption{All HMS Carbon-12 fit with a weighted linear regression using least squares.}
  \label{fig:3-4_hms_linear_regress}
\end{Mfigure}

\begin{Mfigure}{SHMS Linear Regression}
  \centering
  \includegraphics[width=0.85\linewidth]{shms_linear_regress.pdf}
  \caption{All SHMS Carbon-12 fit with a weighted linear regression using least squares.}
  \label{fig:3-4_shms_linear_regress}
\end{Mfigure}

\begin{Mfigure}{HMS Linear Regression LH2}
  \centering
  \includegraphics[width=0.85\linewidth]{hms_linear_regress_lh2.pdf}
  \caption{All HMS LH2 fit with a weighted linear regression using least squares. With this fit there is a slope of $~8\pm2\%$ which is consistent with other predictions of boiling.}
  \label{fig:3-4_hms_linear_regress_lh2}
\end{Mfigure}

\begin{Mfigure}{Boiling Correction}
  \centering
  \includegraphics[width=0.85\linewidth]{boil_eff.pdf}
  \caption{Boiling correction for the highest beam energy setting, $10.6 \mathrm{GeV}$.}
  \label{fig:3-4_boil_eff}
\end{Mfigure}

\begin{figure}
  \centering
  \begin{minipage}[b]{0.48\linewidth}
    \includegraphics[width=\linewidth]{heep_w.pdf}
    %\subcaption{Data without hole cut}
  \end{minipage}
  \hfill
  \begin{minipage}[b]{0.48\linewidth}
    \includegraphics[width=\linewidth]{heep_em.pdf}
    %\subcaption{Data with hole cut}
  \end{minipage}
  
  \vspace{0.5cm}
  
  \begin{minipage}[b]{0.48\linewidth}
    \includegraphics[width=\linewidth]{heep_pmx.pdf}
    %\subcaption{SIMC without hole cut}
  \end{minipage}
  \hfill
  \begin{minipage}[b]{0.48\linewidth}
    \includegraphics[width=\linewidth]{heep_pmy.pdf}
    %\subcaption{SIMC with hole cut}
  \end{minipage}
  \hfill
  \begin{minipage}[b]{0.48\linewidth}
    \includegraphics[width=\linewidth]{heep_pmz.pdf}
    %\subcaption{Data with hole cut}
  \end{minipage}  
  
  \caption{Example kinematic plots comparing HeeP data to SIMC for a beam energy of $4.9 \mathrm{GeV}$.}
  \label{fig:3-4_heep}
\end{figure}

%%%%%%%%%%%%%%%%%%%%%%%%%%%%%%%%%%%%%%%%%%%%%%%%%%%%%%%%%%%%%%%%%%%%%%%%%%%%%%%%%%%%%%%%%%%%%%%%%%%%%%%%%%%%%%%%%%%%%%%%
%
% Chapter 4, section 5
%
%%%%%%%%%%%%%%%%%%%%%%%%%%%%%%%%%%%%%%%%%%%%%%%%%%%%%%%%%%%%%%%%%%%%%%%%%%%%%%%%%%%%%%%%%%%%%%%%%%%%%%%%%%%%%%%%%%%%%%%%

\begin{Mtable}{Target Loop}
  \centering
  \begin{tabular}{|c|c|}
    \hline
    \textbf{Decay Channel} & \textbf{Branching Fracion} \\
    \hline
    $K^+\rightarrow \mu^++\nu_{\mu}$ & 63.55\% \\
    $K^+\rightarrow \pi^++\pi^0$ & 20.66\% \\
    $K^+\rightarrow \pi^++\pi^++\pi^0$ & 5.59\% \\
    $K^+\rightarrow e+\pi^0+\nu_e$ & 5.07\% \\
    $K^+\rightarrow \mu^++\pi^0+\nu_{\mu}$ & 3.35\% \\
    $K^+\rightarrow \pi^++\pi^0+\pi^0$ & 1.76\% \\
    \hline
    \end{tabular}
  \caption{Most probable decay channels of $K^+$ and their respective branching fractions. There are many decay channels for the kaon but these make up >99.98\% of the kaon decay's phase space.}
  \label{tab:4-5_kaon_decay}
\end{Mtable}

%%%%%%%%%%%%%%%%%%%%%%%%%%%%%%%%%%%%%%%%%%%%%%%%%%%%%%%%%%%%%%%%%%%%%%%%%%%%%%%%%%%%%%%%%%%%%%%%%%%%%%%%%%%%%%%%%%%%%%%%
%
% Chapter 5, section 1
%
%%%%%%%%%%%%%%%%%%%%%%%%%%%%%%%%%%%%%%%%%%%%%%%%%%%%%%%%%%%%%%%%%%%%%%%%%%%%%%%%%%%%%%%%%%%%%%%%%%%%%%%%%%%%%%%%%%%%%%%%

\begin{figure}
  \centering
  \begin{minipage}[b]{0.48\linewidth}
    \includegraphics[width=\linewidth]{polar_hi.pdf}
    \subcaption{High $\epsilon$}
  \end{minipage}
  \hfill
  \begin{minipage}[b]{0.48\linewidth}
    \includegraphics[width=\linewidth]{polar_lo.pdf}
    \subcaption{Low $\epsilon$}
  \end{minipage}
  
  \caption{Polar plots of high and low $\epsilon$ for $Q^2=4.4$ and $W=2.74$. The radius is in terms of $-t$.}
  \label{fig:7-1_polar}
\end{figure}

\begin{Mfigure}{Ratios}
  \centering
  \includegraphics[width=0.85\linewidth]{ratios.pdf}
  \caption{Comparison of data to SIMC yield ratios for $Q^2=3.0$, $W=3.14$ for the lowest $t$-bin ($-t=0.215$).}
  \label{fig:7-1_ratios}
\end{Mfigure}

\begin{Mfigure}{Unsep xsect data}
  \centering
  \includegraphics[width=0.85\linewidth]{unsep_data.pdf}
  \caption{Unseparated cross section for $Q^2=3.0$, $W=3.14$ for the lowest $t$-bin ($-t=0.215$). The fit is the unseparated cross section of eqn. \ref{eq:unsep_xsect} which simultaneously fits L, T, LT, and TT.}
  \label{fig:7-1_unsep_data}
\end{Mfigure}

\begin{Mtable}{Parameter Table}
  \centering
  \begin{tabular}{|c|c|c|}
    \hline
    \textbf{Parameter} & \textbf{Initial} & \textbf{Final} \\
    \hline    
    p1 &  0.88669e+03 &   5.83357e+02 \\
    p2 & -0.41000e+03 &  -4.10000e+02 \\
    p3 & -0.25327e+02 &  -2.53270e+01 \\
    p4 &  0.11100e+02 &   1.27321e+01 \\
    p5 &  2.50000e+01 &   1.28000e+01 \\
    p6 &  0.53000e+00 &   9.15840e-01 \\
    p7 &  0.00000e+00 &  -2.42685e-01 \\
    p8 &  0.00000e+00 &   4.93250e-01 \\
    p9 &  0.00000e+00 &   6.09811e-02 \\
    p10 & 0.00000e+00 &  -1.06917e+01 \\
    \hline
    \end{tabular}
  \caption{Result of iterated fit parameters from eqns \ref{eq:model_sig_l}-\ref{eq:model_sig_tt}. Convergence was achieved in 3 iterations.}
  \label{tab:7-1_sig_params}
\end{Mtable}

\begin{figure}
  \centering
  \begin{minipage}[b]{0.48\linewidth}
    \includegraphics[width=\linewidth]{stat_error_highe.pdf}
    \subcaption{High $\epsilon$}
  \end{minipage}
  \hfill
  \begin{minipage}[b]{0.48\linewidth}
    \includegraphics[width=\linewidth]{stat_error_lowe.pdf}
    \subcaption{Low $\epsilon$}
  \end{minipage}
  
  \caption{Run by run systematic uncertainties for $Q^2=3.0$ and $W=2.32$ at high and low $\epsilon$.}
  \label{fig:7-1_stat_error_data}
\end{figure}

\begin{Mtable}{Systematics}
  \centering
  \begin{tabular}{|c|c|c|c|}
    \hline
    \textbf{Source} & \textbf{pt-to-pt (\%)} & \textbf{t-correlated (\%)} & \textbf{scale (\%)} \\
    \hline    
    Acceptance            & 0.4  & 0.4  & 1.0 \\
    PID                   & 1.0  & 0.4  & 0.5 \\
    Coincidence Blocking  &      & 0.2  &     \\
    Tracking Efficiency   & 0.1  & 0.1  & 1.5 \\
    Charge                &      & 0.2  & 0.5 \\
    Target Thickness      &      & 0.2  & 0.8 \\
    Kinematics            & 0.4  & 1.0  &     \\
    Kaon Absorption       &      & 0.5  & 0.5 \\
    Kaon Decay            &      & 1.0  & 3.0 \\
    Radiative Corrections & 0.1  & 0.4  & 2.0 \\
    Monte Carlo Model     & 0.2  & 1.0  & 0.5 \\
    \hline    
    \textbf{Total}        & 1.6  & 2.0  & 4.2 \\
    \hline
    \end{tabular}
  \caption{Modified version of the PAC 34 proposed systematic uncertainties. Thse systematic uncertainties were  estimated based on prior Hall C experiments. One addition to the table is the inclusion of a PID pt-to-pt systematic to take into account the hole in the SHMS HGCer.}
  \label{tab:7-1_pac_error}
\end{Mtable}

%%%%%%%%%%%%%%%%%%%%%%%%%%%%%%%%%%%%%%%%%%%%%%%%%%%%%%%%%%%%%%%%%%%%%%%%%%%%%%%%%%%%%%%%%%%%%%%%%%%%%%%%%%%%%%%%%%%%%%%%
%
% Chapter 7, section 0
%
%%%%%%%%%%%%%%%%%%%%%%%%%%%%%%%%%%%%%%%%%%%%%%%%%%%%%%%%%%%%%%%%%%%%%%%%%%%%%%%%%%%%%%%%%%%%%%%%%%%%%%%%%%%%%%%%%%%%%%%%


\begin{Mfigure}{Diamond Cuts}
  \centering
  \includegraphics[width=0.85\linewidth]{diamond.pdf}
  \caption{Diamond cuts are the result of phase-space matching between high and low $\epsilon$ that show the kinematically accessible region. This is an example from $Q^2=3.0$, $W=3.23$. The red diamond is low and the blue is the high, corrsponding to $\epsilon=0.5736$ and $\epsilon=0.5736$, respectively.}
  \label{fig:7-1_diamond}
\end{Mfigure}

%%%%%%%%%%%%%%%%%%%%%%%%%%%%%%%%%%%%%%%%%%%%%%%%%%%%%%%%%%%%%%%%%%%%%%%%%%%%%%%%%%%%%%%%%%%%%%%%%%%%%%%%%%%%%%%%%%%%%%%%
%
% Chapter 9, section 0
%
%%%%%%%%%%%%%%%%%%%%%%%%%%%%%%%%%%%%%%%%%%%%%%%%%%%%%%%%%%%%%%%%%%%%%%%%%%%%%%%%%%%%%%%%%%%%%%%%%%%%%%%%%%%%%%%%%%%%%%%%

\begin{Mtable}{Kaon Statistics Table}
  \centering
  \begin{tabular}{|c|c|c|c|c|}
    \hline
    \textbf{E (GeV)} & \textbf{$Q^2$ ($GeV^2$)} & \textbf{W (GeV)} & \textbf{$\epsilon$} & \textbf{Total $K^+\Lambda$ Statistics Goal}\\
    \hline
    10.6 & 5.5 & 3.02 & High & ~61\% \\
    8.2 & 5.5 & 3.02 & Low & ~191\% \\
    10.6 & 4.4 & 2.74 & High & ~66\% \\
    8.2 & 4.4 & 2.74 & Low & ~66\% \\
    10.6 & 3.0 & 3.14 & High & ~64\% \\
    8.2 & 3.0 & 3.14 & Low & ~140\% \\
    10.6 & 3.0 & 2.32 & High & 65\%\textbf{*}\tnote{a} \\
    6.2 & 3.0 & 2.32 & Low & ~102\% \\
    10.6 & 2.115 & 2.95 & High & ~42\% \\
    6.2 & 2.115 & 2.95 & Low & ~45\% \\
    \hline
  \end{tabular}
    \begin{tablenotes}
    \item[a] \textbf{*}Only right setting. Center and left were >95\%
    \end{tablenotes}  
  \caption{Summary of $K^+\Lambda$ statistics goal for all settings in the KaonLT 2018-19 experiment.}
  \label{tab:9-1_stat_table}
\end{Mtable}

\begin{Mfigure}{EIC}
  \centering
  \includegraphics[width=0.85\linewidth]{eic.pdf}
  \caption{Planned layout for Electron Ion Collider (EIC) at Brookhaven National Lab \cite{khalek_science_2021}.}
  \label{fig:7-0_eic}
\end{Mfigure}

\begin{table}[]
\tiny
\begin{tabular}{p{4.5cm} p{4cm} p{6.9cm}}   %JRA: could probably make center column narrower, widen others
                                            % CA: I rearranged it as suggested, it saved a couple of lines
                                            % and I tried to keep some symmetry in the last column
\hline
\hline
\textbf{Science Question}  & \textbf{Key Measurement{[}1{]}}    & \textbf{Key Requirements{[}2{]}}   \\ \hline
\multirow{3}{\linewidth}{What are the quark and gluon energy contributions to the pion mass?}                                                   & \multirow{3}{\linewidth}{Pion structure function data over a range of $x$ and $Q^2$.}                      & \multirow{3}{\linewidth}{\begin{tabular}[c]{p{6.7cm}}$\bullet$ Need to uniquely determine e + p $\rightarrow$ e’ + X + n (low -t)\\$\bullet$ CM energy range $\sim$10-100 GeV\\ $\bullet$ Charged- and neutral currents desirable\end{tabular}} \\
 &  & \\
 &  & \\ \hline
\multirow{2}{\linewidth}{Is the pion full or empty of gluons as viewed at large $Q^2$?}                                                            & \multirow{2}{\linewidth}{Pion structure function data at large $Q^2$.}                                       & \multirow{2}{\linewidth}{\begin{tabular}[c]{p{4,3cm}}$\bullet$ CM energy $\sim$100 GeV\\$\bullet$ Inclusive and open-charm detection\end{tabular}}  \\
&  & \\ \hline
\multirow{3}{\linewidth}{What are the quark and gluon energy contributions to the kaon mass?}                                                   & \multirow{3}{\linewidth}{Kaon structure function data over a range of $x$ and $Q^2$.}                      & \multirow{3}{\linewidth}{\begin{tabular}[c]{p{5cm}}$\bullet$ Need to uniquely determine $\Lambda$, $\Sigma^0$:  e + p  $\rightarrow$ e’ + X + $\Lambda /\Sigma^0$ (low -t)\\$\bullet$ CM energy range $\sim$10-100 GeV\end{tabular}}                                                                            \\
 &  & \\
 &  & \\ \hline
\multirow{2}{\linewidth}{Are there more or less gluons in kaons than in pions as viewed at large Q$^2$?}                                           & \multirow{2}{\linewidth}{Kaon structure function data at large $Q^2$.}                                        & \multirow{2}{\linewidth}{\begin{tabular}[c]{p{5cm}}$\bullet$ CM energy $\sim$100 GeV\\$\bullet$ Inclusive and open-charm detection\end{tabular}}                                                               \\
 &  & \\ \hline
\multirow{4}{\linewidth}{Can we get quantitative guidance on the emergent pion mass mechanism?}                                                    & \multirow{4}{\linewidth}{\makecell[l]{Pion form factor data \\for $Q^2$ = 10-40 (GeV/c)$^2$.}}                & \multirow{4}{\linewidth}{\begin{tabular}[c]{p{6cm}}$\bullet$ Need to uniquely determine exclusive process e + p  $\rightarrow$ e’ + p + n (low -t)\\$\bullet$ e-p and e-d at similar energies\\$\bullet$ CM energy $\sim$10-75 GeV\end{tabular}}                                          \\
 &  & \\
 &  & \\
 &  & \\ \hline
\multirow{4}{\linewidth}{What is the size and range of interference between emergent-mass and the Higgs-mass mechanism?}                        & \multirow{4}{\linewidth}{\makecell[l]{Kaon form factor data \\for $Q^2$ = 10-20 (GeV/c)$^2$.}}               & \multirow{4}{\linewidth}{\begin{tabular}[c]{p{6cm}}$\bullet$ Need to uniquely determine exclusive process e + p  $\rightarrow$ e’ + K + $\Lambda$ (low -t)\\$\bullet$  L/T separation at CM energy $\sim$10-20 GeV\\$\bullet$  e-p  $\Lambda /\Sigma^0$ ratios at CM energy $\sim$10-50 GeV\end{tabular}}            \\
 &  & \\
 &  & \\
 &  & \\ \hline
\multirow{3}{\linewidth}{What is the difference between the impacts of emergent- and Higgs-mass mechanisms on light-quark behavior?}            & \multirow{3}{\linewidth}{Behavior of (valence) up quarks in pion and kaon at large $x$}                     & \multirow{3}{\linewidth}{\begin{tabular}[c]{p{6cm}}$\bullet$  CM energy $\sim$20 GeV (lowest CM energy to access large-x region)\\$\bullet$  Higher CM energy for range in $Q^2$ desirable\end{tabular}} \\
 &  & \\
 &  & \\ \hline
\multirow{3}{\linewidth}{What is the relationship between dynamically chiral symmetry breaking and confinement?}                     & \multirow{3}{\linewidth}{Transverse-momentum dependent Fragmentation Functions of quarks into pions and kaons} 
                                    & \multirow{3}{\linewidth}{\begin{tabular}[c]{p{5cm}}$\bullet$ Collider kinematics desirable (as compared to fixed-target kinematics)\\$\bullet$  CM energy range $\sim$20-140 GeV\end{tabular}}\\                   
& & \\
& & \\ 
%\hline
\specialrule{.1em}{.05em}{.05em}
\multicolumn{3}{l}{\textbf{More speculative observables}}                                                      \\ 
\hline
%\specialrule{.1em}{.05em}{.05em}
\multirow{4}{\linewidth}{What is the trace anomaly contribution to the pion mass?}                                                              & \multirow{4}{\linewidth}{Elastic J/$\psi$ production at low W off the pion.}                                  & \multirow{4}{\linewidth}{\begin{tabular}[c]{p{6cm}}$\bullet$ Need to uniquely determine exclusive process e + p  $\rightarrow$ e’ + p + J/$\Psi$ + n (low -t)\\$\bullet$  High luminosity (10$^{34+}$)\\$\bullet$  CM energy $\sim$70 GeV\end{tabular}}                                               \\
& & \\
& & \\
& & \\ \hline
\multirow{4}{\linewidth}{Can we obtain tomographic snapshots of the pion in the transverse plane? What is the pressure distribution in a pion?} & \multirow{4}{\linewidth}{Measurement of DVCS off pion target as defined with Sullivan process}                 & \multirow{4}{\linewidth}{\begin{tabular}[c]{p{6cm}}$\bullet$ Need to uniquely determine exclusive process e + p  $\rightarrow$ e’ + p + g + n (low -t)\\$\bullet$  High luminosity (10$^{34+}$)\\$\bullet$  CM energy $\sim$10-100 GeV\end{tabular}}                                             \\
& & \\
& & \\
& & \\ \hline
\multirow{5}{\linewidth}{Are transverse momentum distributions universal in pions and protons?}                                                 & \multirow{5}{\linewidth}{Hadron multiplicities in SIDIS off a pion target as defined with Sullivan process}    & \multirow{5}{\linewidth}{\begin{tabular}[c]{p{6.3cm}}$\bullet$ Need to uniquely determine scattered off pion: e + p $\rightarrow$ e + h + X + n (low -t)\\$\bullet$  High luminosity (10$^{34}$)\\$\bullet$ e-p and e-d at similar energies desirable\\$\bullet$  CM energy $\sim$10-100 GeV\end{tabular}} \\
& & \\
& & \\
& & \\
& & \\ 
\hline
\hline
\end{tabular}%
%}
%\end{tabularx}
\caption{Science questions related to pion and kaon structure and understanding of the EHM possible accessible at an EIC \cite{khalek_science_2021}.}
\label{tab:EIC_science_questions_table}
\end{table}

\begin{Mfigure}{Cross Section Ratio}
  \centering
  \includegraphics[width=0.85\linewidth]{cross_section_ratio.pdf}
  \caption{Ratios of the uncertainty of the differential cross section from the global fit including EIC projected data to the uncertainty of that without the EIC for various $Q^2$, where (\emph{left panel}) $x=0.001$, (\emph{middle panel}) $x=0.01$, and (\emph{right panel}) $x=0.1$.  For all calculations, $x_L=0.85$.}
  \label{fig:cross_section_ratio}
\end{Mfigure}

\begin{Mfigure}{F2pi}
  \centering
  \includegraphics[width=0.85\linewidth]{F2pi.pdf}
  \caption{Ratio of the uncertainty of the pion structure function from the global fit with and without including EIC projected data for various $Q^2$ values.}
  \label{fig:F2pi}
\end{Mfigure}

\begin{Mfigure}{MC_fpi_logxpi_10on135}
  \centering
  \includegraphics[width=0.85\linewidth]{MC_fpi_logxpi_10on135.pdf}
  \caption{Monte Carlo projections for the pion structure function vs $x$ at a beam energy of 10$\times$135. In this projection, the data was binned in $x$ and $Q^2$, with bin sizes of 0.001 and 10\,$\mathrm{GeV}^2$, respectively. The Monte Carlo projections for $Q^2$ values of 60, 120, 240, 480\,GeV$^2/c$ are represented by dark blue points. For a luminosity of 100\,$\mathrm{fb}^{-1}$, the statistical uncertainties are represented by light blue bands.}
  \label{fig:MC_fpi_logxpi_10on135}
\end{Mfigure}

\begin{Mfigure}{JAM_impact_lesspoints}
  \centering
  \includegraphics[width=0.85\linewidth]{JAM_impact_lesspoints.pdf}
  \caption{\emph{Left:} Comparison of uncertainties on the pion's valence, sea quark, and gluon PDFs before (yellow bands) and after (red bands) inclusion of EIC data.
    \emph{Right:} Ratio of uncertainties with EIC data to without, $\delta^{\rm EIC}/\delta$, for the valence (green line), sea quark (blue), and gluon (red) PDFs. A 1.2\% experimental systematic uncertainty was assumed (no model systematic uncertainty).}
  \label{fig:JAM_impact_lesspoints}
\end{Mfigure}

\begin{Mfigure}{reco-Meson-FF-setup}
  \centering
  \includegraphics[width=0.85\linewidth]{reco-Meson-FF-setup.pdf}
  \caption{Diagram of the Far-Forward beam line and detector setup.}
  \label{fig:reco-Meson-FF-setup}
\end{Mfigure}

\begin{Mfigure}{TDIS_kin_scat_plot}
  \centering
  \includegraphics[width=0.85\linewidth]{TDIS_kin_scat_plot.pdf}
  \caption{A comparison of the scattered electron (left) and leading neutron (right) kinematics for energy settings of $10\times135$ (bottom) and $5\times41$ (top) with momentum, P, and angle, $\theta$, defined in the lab frame. }
  \label{fig:TDIS_kin_scat_plot}
\end{Mfigure}

\begin{figure}
  \centering
  \begin{minipage}[b]{0.48\linewidth}
    \includegraphics[width=\linewidth]{b0_all_IP6.pdf}
  \end{minipage}
  \begin{minipage}[b]{0.48\linewidth}
    \includegraphics[width=\linewidth]{zdc_all_IP6.pdf}
  \end{minipage}
  
  \caption{Top plots: B0 occupancy of the simulated leading neutron at energy settings $5\times41$ (left) and $10\times100$ (right). Bottom plots: ZDC acceptance of the simulated leading neutron at energy settings $5\times41$ (left) and $10\times100$ (right).}
  \label{fig:b0_zdc_all}
\end{figure}


\begin{figure}
  \centering
  \begin{minipage}[b]{0.48\linewidth}
    \includegraphics[width=\linewidth]{delta_t_t_Q2_5on41_IP6.pdf}
  \end{minipage}  
  \begin{minipage}[b]{0.48\linewidth}
    \includegraphics[width=\linewidth]{delta_t_t_Q2_10on100_IP6.pdf}
  \end{minipage}
  
  \caption{The deviation of generated $-t$ from the detected ($t_{truth}$) value, $\Delta{t}=t-t_{truth}$, for a range of energies $5\times41$ and $10\times100$. $Q^2$ is binned in 7, 15, 30, 60 $\mathrm{GeV}^2$ of bin width $\pm5\mathrm{GeV}^2$}
  \label{fig:delta_t_t_Q2_IP6}
\end{figure}
